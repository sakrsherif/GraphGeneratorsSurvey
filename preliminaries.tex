\section{Preliminaries}
\label{sec:preliminaries}


A \emph{graph} $G = (V, E)$, is a set $V$ of $n$ nodes, and a set $E$ of $m$ edges between them. The edges may be undirected or directed. \emph{Bipartite graphs} have edges between two sets of nodes.

...

While the Gaussian distribution is common in nature, there are many cases (such as the WWW, the Internet, citation graph etc.) where the probability of events far to the right of the mean is significantly higher~\cite{Chakrabarti:2006:GML:1132952.1132954}. Power-law distributions attempt to model this. Two variables $x$ and $y$ are related by a \emph{power-law} when $y(x) = A.x^{-k}$, where $A$ and $k$ are positive constants. $k$ is called the power law exponent.

Another important feature of many types of graphs (e.g., typically social networks) is the \emph{community effect}, where a \emph{community} is a set of nodes where each node is closer to the other nodes within the community than to nodes outside it. Communities reveal how a network is internally organized, and indicate the presence of special relationships between the nodes.

Most real-world graphs also have surprisingly small \emph{diameters}, i.e. the longest shortest path between any two graph vertices of a graph. 

...