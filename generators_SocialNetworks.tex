\subsection{Social Networks}
\label{sec:generators_socialnetworks}

On-line social networks, like Facebook, Twitter, or LinkedIn, have become a
phenomenon used by billions of people every day and thus providing extremely
useful information for various domains. However, an analysis of such type of
graphs has to cope with two problems: (1) availability of the data and (2)
privacy of the data. Hence, data generators which provide realistic synthetic
social network graphs are in a great demand.

In general, any analysis of social networks identifies their various specific
features~\cite{Chakrabarti:2006:GML:1132952.1132954}. For example, a social
networks graph often has a high  degree of
transitivity of the graph (so-called \emph{clustering coefficient}). Or, its diameter, i.e., the longest shortest path
amongst some fraction (e.g. 90\%) of all connected nodes, is usually low due to
weak ties joining faraway cliques.

Another key aspect of social networks is the community effect. A detailed
study of structure of communities in 70 real-world networks is provided, e.g.,
in~\cite{Leskovec:2008:SPC:1367497.1367591}.
~\cite{Prat-Perez:2014:CSS:2621934.2621942} analyzed the structure of
communities (clustering coefficient, triangle participation ratio, 
diameter, bridges, conductance, and size) in both real-world graphs and outputs of
existing graph generators such as LFR~\cite{PhysRevE.78.046110} and the
LDBC SNB~\cite{Erling:2015:LSN:2723372.2742786}. They found out that discovered
communities  in different graphs have common distributions and that communities
of a single graph have different characteristics and are challenging to be represented using a single
model.

The existing social network generators try to reproduce different aspects of the
generated network. They can be categorized into statistical and agent-based.
\emph{Statistical
approaches}~\cite{PhysRevE.78.046110,Yao2011,Armstrong:2013:LDB:2463676.2465296,Pham2013,Sukthankar-SocialInfo2014,Erling:2015:LSN:2723372.2742786,Nettleton2016}
focused on reproducing aspects of the network. In \emph{agent-based
approaches}~\cite{Barrett:2009:GAL:1995456.1995598,Bernstein:2013:SAS:2499604.2499609}
the networks are constructed by directly simulating the agents' social choices.

%\paragraph{LFR} Lancichinetti, Fortnato and Radicchi (hence
%LFR)~\cite{PhysRevE.78.046110} develop a class of benchmark graphs whose nodes
%participate in internal community structures. The benchmark models directed and
%weighted real-world networks (e.g., social networks) containing overlapping
%communities of different sizes. The algorithm assumes that both the degree and
%the community size distributions are power laws. Each node shares a fraction $(1
%- \mu)$ of its links with the other nodes of its community and a fraction $\mu$
%with the other nodes of the network, where $\mu$ is called \emph{mixing
%parameter}. The sizes of the communities are taken from a power law distribution
%such that the sum of all sizes equals the number of nodes of the graph. The
%generation process starts with an empty graph and incrementally fills in the
%adjacency matrix by obeying the described constraints.

\paragraph{Realistic Social Network}
~\cite{Barrett:2009:GAL:1995456.1995598} focused on the construction of
realistic social networks. For this purpose the authors combine both
private and public data sets with large-scale agent-based techniques. The process works as
follows: In the first step it generates  synthetic data by combining public and
commercial databases. In the second step, it determines a set of activity
templates. A 24-hour activity sequence including geolocations is assigned to
each synthetic individual. To demonstrate the approach, the authors create a
synthetic US population consisting of people and households together with
respective geolocations. For this purpose the authors combine simulation and
data fusion techniques utilizing various real-world data sources such as U.S.
Census data, responses to a time-use survey or an activity survey.
%Demographic information for each person and location, a minute-by-minute
%schedule of each person's activities, and the locations where these activities
%take place is generated by a combination of simulation and data fusion
%techniques.
The result is captured by a dynamic network of social contacts. Similar methods for
agent-based strategies have been reported
in~\cite{Bernstein:2013:SAS:2499604.2499609}.

\paragraph{Linkage vs. Activity Graphs}
\cite{Yao2011} distinguished between two types of social network graphs -- the
\emph{linkage graph}, where nodes correspond to people and edges correspond to their
friendships, and the \emph{activity graph}, where nodes also represent people
but edges correspond to their interactions. On the basis of the analysis of
Flickr\footnote{\url{https://www.flickr.com/}} social links and
Epinions\footnote{\url{http://www.epinions.com/}} network of user interactions,
the authors discover that they both exhibit high clustering coefficient
(community structure), power-law degree distribution and small diameter.
Considering the dynamic properties they both have relatively stable clustering
coefficient over time and follow the densification law. On the other hand,
diameter shrinking is not observed in Epinions activity graph and there is a
difference in degree correlation (i.e., frequency of mutual connections of
similar nodes) -- activity graphs have neutral, whereas linkage graphs have positive degree correlation. With regards to the findings, the
proposed generator focusses on linkage graphs with positive degree correlation.
For this purpose it extends the forest fire spreading process
algorithm~\cite{Leskovec:2005:GOT:1081870.1081893} with link symmetry. It has
two parameters: the \emph{symmetry
probability} $P_s$ and the \emph{burning probability} $P_b$. $P_b$ ensures a forward burning process based on BFS in which
fire burns strongly with $P_b$ approaching 1.  $P_s$ ensures backward linking
from old nodes to new nodes and ``adds fuel to the fire as it brings more
links''. %It gives chances for big nodes to connect back to big nodes.


\paragraph{LinkBench} The LinkBench
benchmark~\cite{Armstrong:2013:LDB:2463676.2465296} has been designed for the
purpose of analysis of efficiency of a database storing Facebook's production
data. The benchmark considers true Big Data and related problems with sharding,
replication etc. The social graph at Facebook comprises objects (nodes with IDs,
version, timestamp and data) and associations (directed edges, pairs of node
IDs, with visibility, timestamp and data). The size of the target graph is the
number of nodes. Graph edges and nodes are generated concurrently during bulk
loading. The space of node IDs is divided into chunks which enable parallel
processing. The edges of the graph are generated in accordance with the results
of analysing  real-world Facebook data (such as outdegree distribution). A
workload corresponding to 10 graph operations (such as insert object, count the
number of associations etc.) and their respective characteristics over the
real-world data is generated for the synthetic data.

\paragraph{S3G2} The Scalable Structure-correlated Social Graph Generator
(S3G2)~\cite{Pham2013} is a general framework which produces a directed labeled
graph whose vertices represent objects having property values. The respective
classes determine the structure of the properties. S3G2 does not aim at
generating near real-world data, but at generating synthetic graphs with a
correlated structure. Hence, the existing data values influence the probability
of choosing a certain property value from a pre-defined dictionary, or the
probability of connecting two nodes. For example, the degree distribution can be
correlated with the properties of a node and thus, e.g., people having many
friend relationships typically post more comments and pictures. The data
generation process starts with generating a number of nodes with property values
generated according to specified property value correlations and then adding
respective edges according to specified correlation dimensions. It has multiple
phases, each focusing on one correlation dimension. Data is generated in a Map
phase corresponding to a pass along one correlation dimension. Then the data are
sorted along the correlation dimension in the following Reduce phase. A
heuristic observation that ``the probability that two nodes are connected is
typically skewed with respect to some similarity between the nodes'' enables to
focus only on sliding window of most probable candidates. The core idea of the
framework is demonstrated using an example of a social network (consisting of
persons and social activities).  The dictionaries for property values are
inspired by DBpedia and provided with 20 property value correlations. The edges
are generated according to 3 correlation dimensions.


\paragraph{SIB} The developers of the Social Network Intelligence BenchMark (SIB)\footnote{\url{https://www.w3.org/wiki/Social_Network_Intelligence_BenchMark}} based the design of their benchmark on the claim that existing benchmarks are limited in reflecting the characteristics of the real RDF databases and are mostly focusing on the relational style aspects. Hence, they proposed a benchmark for  query evaluation using real graphs~\cite{sib}. The proposed benchmark mimics using an RDF store for a social network. \iffalse site where users and their interactions form a social graph of social activities such as creating/managing groups, writing posts and posting comments.\fi The distribution of the generated data for each type follows the  distribution of the associated type inferred from real-world social networks. Additionally, association rules are exploited for representing the real-world data correlation in the generated synthetic data. The  generated data is linked with the RDF datasets from DBpedia. The benchmark specification contains 3 query mixes -- interactive, update, and analysis -- expressed in SPARQL 1.1 Working Draft.

\paragraph{Cloning of Social Networks} ~\cite{Sukthankar-SocialInfo2014}
introduces two synthetic generators to reproduce two characteristics typically
observed in social networks: node features and multiple link types. Both
generators extend  the generator proposed by~\cite{wang2011leveraging}.
which starts with a small number of
nodes and new nodes are added until the network reaches the required number. It
has two basic parameters: homophily and link density. A high \emph{homophily}
value reflects that links have higher chances to be established among the nodes with the
same label; these labels can be considered as being equivalent to community
membership.

The first proposed generator is Attribute Synthetic Generator (ASG), used for
reproducing the node feature distribution of standard networks and rewiring the
network to preferentially connect nodes that exhibit a high feature similarity.
The network is initialized with a group of three nodes. New nodes and links
are added to the network based on link density, homophily, and feature
similarity. As new nodes are created, their labels are assigned based on the
prior label distribution. After the network has reached the same number of nodes
as the original social media dataset, each node initially receives a random
attribute assignment. Then a stochastic optimization process is used to move the
initial assignments closer to the target distribution extracted from social
media dataset using the Particle Swarm Optimization algorithm. The tuned
attributes are then used to add additional links to the network based on the
feature similarity parameter -- a source node is selected randomly and connected
to the most similar node. The second proposed generator, so-called Multi-Link Generator
(MLG), further  uses link co-occurrence statistics from the original dataset to
create a multiplex network. MLG uses the same network growth process as ASG.
Based on the link density parameter, either a new node is generated with a label
based on the label distribution of the target dataset or a new link is created
between two existing nodes.


\paragraph{LDBC SNB} Despite having a common Facebook-like dataset, thanks to
three distinct workloads the Social Network Benchmark
(SNB)~\cite{Erling:2015:LSN:2723372.2742786} provided by LDBC represents three
distinct benchmarks. The network nodes correspond to people and the edges
represent their friendship and messages they post in discussion trees on their
forums. The three query workloads involve: (1) SNB-Interactive, i.e., complex
read-only queries accessing a significant portion of data, (2) SNB-BI, i.e.,
queries accessing a high percentage of  entities and grouping them in various
dimensions, and (3) SNB-Algorithms, involving graph analysis algorithms, such as
community detection, PageRank, BFS, and clustering. The graph generator, called
Datagen, is a fork of S3G2 ~\cite{Pham2013} and realizes power laws, uses skewed
value distributions, and ensures reasonable correlations between graph
structures and property values. Additionally, it extends S3G2 with "spiky"
patterns in the distribution of social network activity along the timeline, also
provides the ability of generating update streams to the social network. Datagen
is also based on Hadoop in order to provide scalability, but comparet to S3G2 it
contains numerous performance improvements and the ability to be deterministic
regardless of the number of computer nodes used for the generation of the graphs
and for a given set of configuration parameters.

%The generated data have become a part if several graph benchmarks, such as GraphBIG~\cite{Nai:2015:GUG:2807591.2807626}.

\paragraph{Towards More Realistic Data} \cite{Nettleton2016} argued that the majority of existing works focuses on
topology generation which approximates the characteristics of a real-world social
network (e.g., a small graph diameter, skew degree
distribution, small average path length, and community structures); however,  this is usually done without any data. Hence, they
introduced a general stochastic modeling approach that enables the users to
populate a graph topology with data. The approach has three steps: (1) topology
generation (using R-MAT) plus community identification using the Louvain
method~\cite{1742-5468-2008-10-P10008} or usage of a real-world topology from
SNAP\footnote{\url{https://snap.stanford.edu/data/}}, (2) data definition
that describes distribution profiles, attribute value definitions, using a
parameterizable set of data propagation rules and affinities, and (3) data
population.

\paragraph{Strengths and Weaknesses of Social Network Generators}
Compared to more general graph generators, social network generators focus
mainly on reproducing intra- and inter-node feature correlations.
Among existing generators, LDBC SNB and S3G2 look
like the most advanced ones in terms of the complexity of the generated graph
and the amount of features and correlations they can generate, while providing a
large degree of scalability. Their generation process is based on input dictionaries and have configuration files that
allow tweaking  many parameters of the generated graphs, but their schema is mainly
static and cannot be easily configured to meet the needs of other uses cases
besides the benchmarks they have been designed for. In this regard, the approaches
like those proposed in~\cite{Nettleton2016} and~\cite{Sukthankar-SocialInfo2014} offer a
more flexible and understandable configuration process to tweak the types,
values, and correlations between different features.

Regarding the correlation between the underlying graph structure and the node
features, approaches such as LDBC SNB, S3G2 or ~\cite{Nettleton2016}
take into account this aspect and the generated graphs have realistic
structural properties while similar nodes have a larger probability of being
connected. However, their approach seems to be more based on
intuition and common sense than to be backed up by  any study of how the
relation between structure and attributes showcase in real social networks. In
this regard, this remains as a clear open challenge for social network
generators.

Finally, scalability is another aspect to be considered in social network graph
generators. LDBC SNB and S3G2 are engineered with this in mind, thus they
provide a way to scale to billions of nodes and edges. This is not the case for
the other generators, which can make them impractical if our goal is to generate
real sized social network graphs.


