\section{Graph Data Generators}
\label{sec:generators}

...



\subsection{General Graphs}
\label{sec:generators_general}

First of all we will focus on papers dealing with generation of general graph data not aiming at particular application domain. Currently, there exists a number of tools which involve a kind of general graph data generator, such as gtools from projects nauty and Traces~\cite{gtools} or the Stanford GraphBase~\cite{GraphBase}. We will, however, focus on primarily generating/benchmarking projects targeting Big Data world.



\paragraph{GraphGen} For the purpose of testing the scalability of an indexing technique called FG-index~\cite{Cheng:2007:FTV:1247480.1247574} on size of the database of graphs, their average size and average density, the authors have also implemented a synthetic generator called GraphGen~\cite{GraphGen}. It is based on the IBM synthetic data generation code for associations and sequential patterns\footnote{From 1996, no longer available at \url{http://www.almaden.ibm.com/cs/projects/iis/hdb/Projects/data mining/mining.shtml}.}. GraphGen creates a collection of labeled, undirected and connected graphs aiming at performance evaluation of frequent subgraph mining algorithms and graph query processing algorithms. The result is represented as a list of graphs, each consisting of a list of nodes and a list of edges.



\paragraph{HPC Scalable Graph Analysis Benchmark} The HPC Scalable Graph Analysis Benchmark~\cite{HPCgraph,Bader:2005:DIH:2099301.2099360} represents an application with multiple analysis techniques that access a single data structure representing a weighted, directed graph. The benchmark is composed of four separated operations (graph construction, classification of large vertex sets, graph extraction with BFS, and graph analysis with betweenness centrality) on a graph that follows a power-law distribution.

The graph generator constructs a list of edge tuples containing vertex identifiers (with the edge direction from the first one to the second one) and weights that represent data assigned to the edges of the multigraph in the form of positive integers with a uniform random distribution. The generator has the following parameters: number of vertices, number of edges, and maximum weight of an edge. The algorithm of the generator is based on R-MAT. Since the authors aim to avoid data locality, in the final step the vertex numbers are randomly permuted, and then edge tuples randomly shuffled.

A related project from the same authors developed for the 9th DIMACS Shortest Paths Challenge is GTgraph~\cite{GTgraph}. It involves three types of graphs: input graph instances used in the DARPA HPCS SSCA\#2 graph theory benchmark (version 1.0), Erd\"{o}s-R\'{e}nyi random graphs, and small-world graphs based on the R-MAT model.


\paragraph{Graph 500 Benchmark} The Graph 500 Benchmark~\cite{Graph500} includes a scalable data generator which produces weighted, undirected graph as a list of edge tuples containing the label of start vertex and end vertex together with a weight that represents data assigned to the edge. The space of vertex  labels is the set of integers beginning with 0. The input values required to describe the graph are (1) scale, i.e., the logarithm base two of the number of vertices, and (2) edge factor, i.e., the ratio of the graph's edge count to its vertex count (i.e., half the average degree of a vertex in the graph). The graph generator is a Kronecker generator similar to R-MAT. The data must not exhibit any locality, so in the final step the vertex labels and order of edges are randomly shuffled.

The covered operations currently involve BFS; however, the authors intend to involve also two more types -- optimization (single source shortest path) and edge-oriented (maximal independent set) -- and five graph-related business areas cybersecurity, medical informatics, data enrichment, social networks, and symbolic networks.


%\paragraph{gtools} Projects nauty and Traces~\cite{gtools} are programs for computing automorphism groups of graphs and digraphs~\cite{McKay201494}; they can also produce a canonical label. In the respective package there is also suite of programs called gtools which involve generators for non-isomorphic graphs, bipartite graphs, digraphs, and multigraphs.





\subsection{Semantic Web}
\label{sec:generators_LinkedData}
With the dawn of the concept of Linked Data it is a natural development that there would emerge respective benchmarks involving both synthetic data and real-world data sets  sets with real-world characteristics. The used data sets correspond to RDF representation of relational-like data~\cite{Guo2005158,Bizer09theberlin}, social network-like data~\cite{Schmidt2010}, or specific and significantly more complex data structures such as biological data~\cite{Wu2014}. In this section, we provide an overview of benchmarking systems involving a kind of graph-based RDF data generator or data modifier. %Other types of systems or particular results can be found, e.g., at~\cite{RdfStoreBenchmarking}.

\iffalse
Considering the Big Data world, the Linked Data in general definitely belong to this group since we assume that the Linked (Open) Data Sets form a common Linked Open Data cloud\footnote{\url{http://lod-cloud.net/}}. On the other hand, the particular data sets can be relatively small.
\fi

\paragraph{LUBM} The use-case driven Lehigh University Benchmark (LUBM)\footnote{\url{http://swat.cse.lehigh.edu/projects/lubm/}} considers the university domain. The ontology defines 43 classes and 32 properties~\cite{Guo2005158}. In addition, the LUBM benchmark provides 14 test queries. In particular, the benchmark focuses on \emph{extensional} queries, i.e., queries which target the instance data over ontologies, as an opposite to \emph{intentional} queries, i.e., queries which target the properties and  classes of the ontology. The Univ-Bench Artificial  (UBA) data generator features repeatable and random  data generation (exploiting classical linear congruential generator, LCG, of numbers). In particular, the data which is produced by the generator are assigned zero-based indexes (i.e., the first university is named \emph{University0} and so on), thus they are reproducible at any time with the same indexes.  The generator allows to specify a seed for random number generation, along with the desired number of universities, and the starting index of the universities.

An extension of LUBM, the Lehigh BibTeX Benchmark (LBBM)~\cite{Wang2005}, enables generating synthetic data for different ontologies. The data generation process is managed through two main phases: (1) the property-discovery phase, and (2) the data generation phase. In particular, LBBM provides a probabilistic model that can emulate the properties of the data of a particular domain and generate synthetic data exhibiting similar properties. A Monte Carlo algorithm is employed to output synthetic data. The approach is demonstrated on the Lehigh University BibTeX ontology which consists of 28 classes along with 80 properties. The LUBM benchmark includes 12 test queries that were designed for the benchmark data. Another extension of LUBM, the University Ontology Benchmark (UOBM)\footnote{\url{https://www.cs.ox.ac.uk/isg/tools/UOBMGenerator/}}, focuses on two aspects: (1) usage of all constructs of OWL Lite and OWL DL~\cite{owl} and (2) lack of necessary links between the generated data which thus form isolated graphs~\cite{Ma:2006:TCO:2094613.2094629}. In the former case the original ontology is replaced by the two types of extended versions from which the user can choose. In the latter case cross-university and cross-department links are added to create a more complex graph.

\paragraph{IIMB} Contrary to the previous work, Ferrara et al.~\cite{Ferrara08OM} proposed the ISLab Instance Matching Benchmark (IIMB)\footnote{\url{http://www.ics.forth.gr/isl/BenchmarksTutorial/}} for the problem of instance matching. For any two objects $o_1$ and $o_2$ adhering to the same ontology or to different ontologies, instance matching is specified in the form of a function $Om(o_1, o_2) \rightarrow \{0, 1\}$,  where $o_1$ and $o_2$ are linked to the same real-world object (in which case the function maps to $1$) or $o_1$ and $o_2$ are representing different objects (in which case the function maps to $0$). It targets the domain of movie data which contains 15 named classes, along with 5 object properties and 13 datatype properties. The data are extracted from IMDb\footnote{\url{http://www.imdb.com/}}. The data generator corresponds to a data modifier which simulates differences between the data. In particular it involves data value differences (such as typographical errors or usage of different standard formats, e.g., for names), structural heterogeneity (represented by different levels of depth for properties, diverse aggregation criteria for properties, or missing values specification) and logical heterogeneity (such as instantiation on different subclasses of the same superclass or instantiation on disjoint classes).


\paragraph{BSBM} The Berlin SPARQL Benchmark (BSBM)\footnote{\url{http://wifo5-03.informatik.uni-mannheim.de/bizer/berlinsparqlbenchmark/}}, is centered around an e-commerce application domain with object types such as \emph{Customer}, \emph{Vendor}, \emph{Product} and \emph{Offer} in addition to the relationship among them~\cite{Bizer09theberlin}.
The benchmark provides a workload that have 12 queries with 2 types of query workloads (i.e., 2 sequences of the 12 queries) emulating the navigation pattern and search of a consumer seeking a product. The data generator is capable of producing arbitrarily scalable datasets by controlling the number of products ($n$) as a scale factor.  The scale factor also impacts other data characteristics, such as, e.g., the depth of type hierarchy of products (defined as $d = round(log_{10}(n))/2 + 1$), branching factor ($bfr = 2 \times round(log_{10}(n))$), the number of product features (having $lowerBound = 35 \times i / (d \times (d+1)/2 - 1)$ and $upperBound = 75 \times i / (d \times (d+1)/2 - 1)$) etc. BSBM can output two representations, i.e. an RDF representation along with a relational representation. Thus, BSBM also defines an SQL~\cite{sql} representation of the queries. This allows comparison of SPARQL~\cite{sparql} results  to be compared against the performance of traditional RDBMSs.


\paragraph{SP$^2$Bench} The SP$^2$Bench\footnote{\url{http://dbis.informatik.uni-freiburg.de/forschung/projekte/SP2B/}} is a language-specific benchmark~\cite{Schmidt2010} which is based on the DBLP dataset. %, so the types involve Person, Inproceedings, Article etc.
The generated datasets follow the key characteristics of the original DBLP dataset. In particular, the data mimics the correlations between entities. All random functions of the generator use a fixed seed that ensures the data generation process is deterministic. SP$^2$Bench is accompanied by 12 queries covering the various types of operators such as RDF access paths in addition to typical RDF constructs.

%\paragraph{JustBench} ~\cite{Bail:2010:JFO:1940281.1940285} ...

\paragraph{Data Coherence}  A comparison of 4 RDF benchmarks (namely TPC-H~\cite{TPC-H} data expressed in RDF, LUBM, BSBM, and SP$^2$Bench) and 6 real-wold data sets (such as, e.g.,  DBpedia, the Barton Libraries Dataset~\cite{barton-benchmark} or
WordNet~\cite{Miller:1995:WLD:219717.219748}) has been reported by~\cite{Duan:2011:AOC:1989323.1989340}. The authors focus mainly on the  \emph{structuredness} (\emph{coherence}) of each benchmark dataset claiming that a primitive metric (e.g., the number of triples or the average in/outdegree) quantifies only some target characteristics of each dataset. The degree of structuredness of a dataset $D$ with respect to a type $T$ is based on  the regularity of instance data in $D$ in conforming to type $T$. The type system is extracted from the data set by finding the RDF triples that have property  \texttt{http://www.w3.org/1999/02/22-rdf-syntax-ns\#type} and extract type $T$ from their object. Properties of $T$ are determined as the union of all the properties that the instances of type $T$ have. The structuredness is then expressed as a weighted sum of share of set properties of each type, whereas higher weights are assigned to types with more instances. The authors show that the structuredness of the chosen benchmarks is fixed, whereas real-world RDF datasets are belonging to the non-tested area of the spectrum. As a consequence, they introduce a new benchmark that receives as input any dataset associated with a required level of structuredness and size (smaller than the size of the original data), and exploits the input documents as a seed to produce a subset of the original data with the target structuredness and size. In addition, they show that structuredness and size mutually influence each other.


\paragraph{DBPSB} DBpedia SPARQL Benchmark (DBPSB)\footnote{\url{http://aksw.org/Projects/DBPSB.html}} proposed at the University of Leipzig has been designed using workloads that have been issued by humans and applications over existing RDF data~\cite{Morsey2011,Morsey:2012:UBR:2900929.2901031}. In addition, the authors argue that benchmarks like LUBM, BSBM, or SP$^2$Bench resemble relational database benchmarks involving relational data structures with few and homogeneously structured classes, whereas, in reality, RDF datasets are increasingly heterogeneous. For example, DBpedia version 3.6 consists of 289,016 classes of which 275 classes are defined based on the DBpedia ontology. In addition, different datatypes and object references of various types are
used in property values. Hence, they presented a universal SPARQL benchmark generation approach which uses a flexible data production mechanism that mimics the input data source. This dataset generation process begins using an input dataset then multiple datasets with different sizes  are then generated by duplicating all the RDF triples with changing their namespaces.  For generating smaller datasets, an appropriate fraction of all triples is selected randomly or by sampling across classes in the dataset. \iffalse The goal of the query analysis and clustering is to detect prototypical queries on the basis of their frequent usage and similarity.\fi The methodology is applied on the DBpedia SPARQL endpoint and a set of 25 SPARQL query templates is derived, that cover the most commonly used SPARQL features.

\paragraph{LODIB} The Linked Open Data Integration Benchmark (LODIB)\footnote{\url{http://lodib.wbsg.de/}} has been designed with the aim of reflecting the real-world heterogeneities that exist on the Web of Data in order to enable testing of Linked Data translation systems~\cite{DBLP:conf/www/RiveroSBR12}. It provides a catalogue of 15 data translation patterns (e.g., rename class, remove language tag etc.), each of which is a common data translation problem in the context of Linked Data. The benchmark provides a data generator that produces three different synthetic data sets that need to be translated
by the system under test into a single target vocabulary. They  reflect the pattern distribution in analyzed 84 data translation examples from the LOD Cloud. The data sets reflect the same e-commerce scenario used for BSBM.


\paragraph{SIB} The developers of the Social Network Intelligence BenchMark (SIB)\footnote{\url{https://www.w3.org/wiki/Social_Network_Intelligence_BenchMark}} based the design of their benchmark based on the claim that existing benchmarks are limited in reflecting the characteristics of the real RDF datasets and are mostly relational-like. Hence, they proposed a benchmark for  query processing over real graphs~\cite{sib}. The proposed benchmark simulates an RDF backend of a social network site where users and their interactions form a social graph of social activities such as creating/managing groups, writing posts and posting comments. The distribution of generated data on each relation follows the data distribution inferred from real-world social networks. In addition, association rules are included in order to convey the real-world data correlation into synthetic data. The  generated data is linked with the RDF datasets from DBpedia. The benchmark specification contains 3 query mixes -- interactive, update, and analysis -- expressed in SPARQL 1.1 Working Draft.

\paragraph{Geographica} The Geographica benchmark\footnote{\url{http://geographica.di.uoa.gr/}} has been designed to target the area of geospatial data~\cite{DBLP:conf/semweb/GarbisKK13} and respective SPARQL extensions GeoSPARQL~\cite{battle2012enabling} and stSPARQL~\cite{koubarakis2010modeling}. The benchmark involves a real-world workload based on openly available datasets that cover a range of geometry types (e.g., points, lines, polygons) and  a synthetic workload. In the former case there is a (1) a micro benchmark that tests primitive spatial functions (involving 29 queries) and (2) macro benchmark that tests the performance of RDF stores in typical application scenarios like reverse geocoding or map search and browsing (consisting of 11 queries). In the latter case of a synthetic workload the generator produces synthetic datasets of various sizes that corresponds to an ontology based on OpenStreetMap (i.e., states in a country, land ownership, roads and  points of interest) and instantiates query templates. The spatial extent of the land ownership dataset constitutes a uniform grid of $n \times n$ hexagons, whereas the size of each dataset is given relatively to $n$. The synthetic workload generator produces SPARQL queries corresponding to spatial selection and spatial joins by instantiating 2 query templates.


\paragraph{WatDiv} The Waterloo SPARQL Diversity Test Suite (WatDiv)\footnote{\url{http://dsg.uwaterloo.ca/watdiv/}} developed at the University of Waterloo provides stress testing tools to address the observation that existing SPARQL benchmarks are not suitable for testing systems for diverse queries and varied workloads~\cite{Aluc:2014:DST:2717213.2717229}. The benchmark introduces two classes of query features -- structural and data-driven -- and performs a detailed analysis on existing SPARQL benchmarks (LUBM, BSBM, SP$^2$Bench, and DBPSB) using the two classes of query features. The structural features involve triple pattern count, join vertex count, join vertex degree, and join vertex count. The data-driven features involve result cardinality and several types of selectivity. The analysis of the four benchmarks reveals that they are not sufficiently diverse to evaluate the strengths and weaknesses of the various physical design alternatives that have been implemented by the different RDF systems. The proposed solution, WatDiv, involves (1) a data generator which generates scalable datasets according to the WatDiv schema, (2) a query template generator which traverses the WatDiv schema and generates a diverse set of query templates, (3) a query generator which instantiates the templates with actual RDF terms from the dataset, and (4) a feature extractor which extracts the structural features of the generated data and workload. %For the study in the paper the authors generated 12,500 test queries from 125 query templates.

\paragraph{RBench} RBench~\cite{Qiao:2015:RAR:2723372.2746479} is an application-specific benchmark which receives any RDF dataset as an input and produces as an output a set of datasets, that have similar characteristics of the input dataset, using size scaling factor $s$ and (node) degree scaling factor $d$. \iffalse A generated benchmark dataset is considered similar to the given dataset if their values for the dataset evaluation metrics and query evaluation times for different techniques are similar. Three evaluation metrics are utilized for this purpose: dataset coherence (i.e., a measure how uniformly predicates are distributed among the same type/class), relationship specialty (i.e., the number of occurrences of the same predicate associated with each resource), and literal diversity.\fi A query generation process has been implemented to produce 5 different types of queries (edge-based queries, node-based queries, path queries, star queries, subgraph queries) for any generated data. The benchmark project FEASIBLE~\cite{Saleem2015} is also an application-specific benchmark; however, contrary to RBench, it is designed to produce benchmarks from a set of queries (in particular from query logs) by relying on sample queries of a user-defined
size from the input set of queries.


\paragraph{LDBC}  The Linked Data Benchmark Council\footnote{\url{http://ldbcouncil.org/industry/organization/origins}} (LDBC)~\cite{Angles:2014:LDB:2627692.2627697} is a result of a (closed) EU project that brought together a community of academic researchers and industry that had the main objective of developing an open source, yet industrial grade benchmarks for graph and RDF databases. \iffalse The following three benchmarks were developed and are currently maintained.\fi In the Semantic Web domain, the project released the Semantic Publishing Benchmark (SPB)~\cite{spb} that has been inspired by the Media/Publishing industry (namely BBC\footnote{\url{http://www.bbc.com/}}). The application scenario of this benchmark simulates a media or a publishing organization that handles large amount of streaming content (e.g., news, articles). \iffalse This content is enriched with metadata that describes it and links it to reference knowledge -- taxonomies and databases that include relevant concepts, entities and factual information. The SPB data generator produces scalable in size synthetic large data. Synthetic data consists of a large number of annotations of media assets that refer entities found in reference datasets.\fi The data generator mimics three types of relations in the generated synthetic data: clustering of data, correlations of entities, and random tagging of entities. Two workloads are provided: (1) basic, involving an interactive query-mix querying the relations between entities in reference data, and (2) advanced,  focusing on interactive and analytical query-mixes. The LDBC has designed two other benchmarks: the Social Network Benchmark (SNB)~\cite{Erling:2015:LSN:2723372.2742786} for the social network domain  (see Section~\ref{sec:generators_socialnetworks}) and Graphalytics~\cite{Iosup:2016:LGB:3007263.3007270}   for the analytics domain.% (see Section~\ref{sec:generators_analytics}).



\paragraph{LinkGen} LinkGen is a synthetic linked data generator that has been designed to generate RDF datasets for a given vocabulary~\cite{10.1007/978-3-319-46547-0_12}. The generator is designed to receive a vocabulary as an input  and supports two statistical distributions for generating entities: Gaussian distribution and Zipf's power-law distribution. LinkGen can augment the generated data with inconsistent and noisy  data such as writing two conflicting values for a given datatype property,  adding triples with syntactic errors, adding wrong statements by assigning them with invalid domain and creating instances with no type information. The generator is also designed with  an option to inter-link the generated instances with real ones given that the user provides entities from real datasets. The datasets can be generated in any of of two modes: on-disk and streaming.



\subsection{Graph Databases}
\label{sec:generators_GraphDatabases}

Currently there exists a number of papers which compare the efficiency of graph databases with regards to distinct use cases, such as  the community detection problem~\cite{Beis2015}, social tagging systems~\cite{Giatsoglou2011}, graph traversal~\cite{Ciglan:2012:BTO:2374486.2375242}, graph pattern matching~\cite{Pobiedina2014}, data provenance~\cite{Vicknair:2010:CGD:1900008.1900067}, or even several distinct use cases~\cite{Grossniklaus2013Towar-24253}. However, the number of graph data generators and benchmarks that have been designed specifically for graph data management systems (Graph DBMS)  is relatively small. Either a general graph generator is used for benchmarking graph databases, such as, e.g., the HPC Scalable Graph Analysis Benchmark~\cite{Dominguez-Sal:2010:SGD:1927585.1927590} or the graph DBMS benchmarking tools are designed while having in mind a more general scope. Hence it is questionable whether a benchmark  that is targeted specifically for graph databases is necessary. \cite{Dominguez-Sal:2010:DDG:1946050.1946053} discussed this question and related topics. On the basis of a review of applications of graph databases (namely, social network analysis,  genetic interactions and recommendation systems), the authors analyzed and discussed the features of the graphs for these types of applications and how such features can affect the benchmarking process, various types of operations used in these applications and the characteristics of the evaluation setup of the benchmark. In this section, we focus on graph data generators and benchmarks that have been primarily targeting graph DBMSs.


\paragraph{XGDBench} XGDBench~\cite{Dayarathna:2014:GDB:2676904.2676939} is an extensible  benchmarking platform for graph databases used in cloud-based systems. Its intent is to automate
the process of graph database benchmarking in the cloud by focusing on the domain social networking services. It extends the Yahoo! Cloud Multiplicative Attribute (MAG) Graph Serving Benchmark (YCSB)~\cite{Cooper:2010:BCS:1807128.1807152} and provides a set of standard workloads representing various performance issues. In particular, the workload of XGDBench involves basic operations such as read / insert / update / delete an attribute, loading of the list of neighbours and BFS traversal. Using the generators, 7 workloads are created, such as update heavy, read mostly, short range scan, traverse heavy etc.
The data model of XGDBench is a simplified version of the Multiplicative Attribute Graph (MAG)~\cite{Kim2010} model, a synthetic graph model which models the interactions between node attributes and  graph structure.
The generated graphs are thus in MAG format, with power-law degree distribution closely simulating real-world social networks.
The simplified MAG algorithm accepts the required number of nodes, the number of attributes per each node, a threshold value for random attribute initialization, an edge affinity threshold determining existence of an edge between two nodes, and an affinity matrix. \iffalse It has been proven that MAG generates graphs with both analytically tractable and statistically interesting properties.\fi
Large graphs can be generated on multi-core systems by a multi-threaded version of the  generator.


\paragraph{gMark}  gMark~\cite{gMark} is a schema-driven and domain-agnostic generator of both graph instances and graph query workloads. It can generate instances under the form of N-triples and queries in various concrete query languages, including OpenCypher\footnote{\url{https://neo4j.com/developer/cypher-query-language/}}, recursive SQL, SPARQL and LogicQL. In gMark, it is possible to specify a \emph{graph configuration} involving the admitted edge predicates and node labels occurring in the graph instance along with additional parameters such as degree distribution, occurrence constraints, etc. The \emph{Query workload configuration} describes parameters of the query workload to be generated, by including the number of queries, arity, shape and selectivity of the queries.
The problem of deciding whether there exists a graph that satisfies a defined graph specification $G$ is NP-complete. The same applies to the problem of deciding
whether there exists a query workload compliant with a given query workload configuration $Q$. In view of this, gMark adopts a best effort approach in which the
parameters specified in the configuration files are attained in a relaxed fashion in order to achieve linear running time whenever possible.

%The authors prove that deciding whether there exists a graph satisfying a given graph configuration $G$ is NP-complete. And, similarly, deciding
%whether there exists a query workload satisfying a given query workload configuration $Q$ is also NP-complete. Hence, gMark generating is based on a heuristic strategy: it tries to achieve the exact values of the given parameters, however, in order to obtain linear running time it may decide to relax some. gMark generates graphs under the form of N-triples and query workloads in four concrete syntaxes, including Cypher\footnote{\url{https://neo4j.com/developer/cypher-query-language/}}, SPARQL, SQL and LogicQL.

\paragraph{GraphGen}  GraphAware GraphGen\footnote{\url{http://graphgen.graphaware.com/}} is a graph generation engine based on Neo4j's\footnote{\url{https://neo4j.com/}} query language OpenCypher~\cite{GraphGen}.  It creates nodes and relationships based on a schema definition expressed in Cypher, and it can also generate property values on both
nodes and edges. As such, GraphGen is a precursor of property graphs generators. The resulting graph can be exported to several formats (namely GraphJson\footnote{\url{https://github.com/GraphAlchemist/GraphJSON/wiki/GraphJSON}} and CypherQueries) or loaded directly to a DBMS. However, it is very likely that it is not maintained anymore due to the lack of available recent commits.

\paragraph{Strengths and Weaknesses of Graph Database Generators}
The graph DBMS generators discussed in this section have in common the fact that they can generate semantically rich labeled graphs with properties (ranging from properties values in GraphGen to MAG structures in XGDBench). They are also capable of generating graph instances and query workloads in
concrete syntaxes (among which OpenCypher in GraphGen and gMark) and one of them (XGDBench) can also handle update operations on both graph structure and content. However, more comprehensive graph DBMS generators that also produce data manipulation operations (such as updates for graph databases) are urgently needed. Additionally, none of these generators is enabled to work on corresponding query languages for property graphs, such as the newly emerging standard GQL~\cite{gql-2018} and G-Core~\cite{AnglesABBFGLPPS18}. Hence, a full-fledged graph DBMS generator for property graphs and property graph query workloads~\cite{BFVY18} is still missing and there exists an interesting opportunity to build such a generator in the near future.

Another apparent inconvenience is represented by the fact that explicit correlations among graph elements cannot be encoded for instance in gMark or GraphGen, whereas they could be fruitful in order to reproduce the behavior of real-world graphs in which attribute values are correlated one with another. On the other hand, social network and Linked Data generators that support correlations (as highlighted in Section \ref{sec:generators_socialnetworks} and Section \ref{sec:generators_LinkedData}) typically exhibit a fixed schema and are not not necessarily multi-domain as are some of the graph DBMS generators discussed in this section (namely GraphGen and gMark).



\subsection{Social Networks}
\label{sec:generators_socialnetworks}

On-line social networks, like Facebook, Twitter, or LinkedIn, have become a
phenomenon used by billions of people every day and thus providing extremely
useful information for various domains. However, an analysis of such type of
graph has to cope with two problems: (1) availability of the data and (2)
privacy of the data. Hence, data generators which provide realistic synthetic
social network graphs are in a great demand.

In general, any analysis of social networks identifies their various specific
features~\cite{Chakrabarti:2006:GML:1132952.1132954}. For example, a social
graph often has high \emph{clustering coefficient}, i.e. the degree of transitivity of
a graph. Or, its diameter, i.e. the longest shortest path amongst some fraction (e.g. 90\%) of all connected nodes, is usually low due to weak ties joining faraway cliques.

Another important aspect
of social networks is the community effect. A detailed study of structure of
communities in 70 real-world networks is provided, e.g., in
~\cite{Leskovec:2008:SPC:1367497.1367591}.
~\cite{Prat-Perez:2014:CSS:2621934.2621942} analyzed the structure of
communities (clustering coefficient, triangle participation ratio, bridges,
diameter, conductance and size) in both real-world graphs and outputs of existing graph
generators LFR~\cite{PhysRevE.78.046110} and the
LDBC-SNB~\cite{Erling:2015:LSN:2723372.2742786}. They discover that communities found in different graphs follow quite similar distributions and that communities in a single graph have diverse nature and are difficult to fit with a single model.

The existing social network generators try to reproduce different aspects of the
generated network. They can be categorized into statistical and agent-based.
\emph{Statistical
approaches}~\cite{PhysRevE.78.046110,Yao2011,Armstrong:2013:LDB:2463676.2465296,Pham2013,Sukthankar-SocialInfo2014,Erling:2015:LSN:2723372.2742786,Nettleton2016}
focused on reproducing aspects of the network. In \emph{agent-based
approaches}~\cite{Barrett:2009:GAL:1995456.1995598,Bernstein:2013:SAS:2499604.2499609}
the networks are constructed by directly simulating the agents' social choices.

%\paragraph{LFR} Lancichinetti, Fortnato and Radicchi (hence
%LFR)~\cite{PhysRevE.78.046110} develop a class of benchmark graphs whose nodes
%participate in internal community structures. The benchmark models directed and
%weighted real-world networks (e.g., social networks) containing overlapping
%communities of different sizes. The algorithm assumes that both the degree and
%the community size distributions are power laws. Each node shares a fraction $(1
%- \mu)$ of its links with the other nodes of its community and a fraction $\mu$
%with the other nodes of the network, where $\mu$ is called \emph{mixing
%parameter}. The sizes of the communities are taken from a power law distribution
%such that the sum of all sizes equals the number of nodes of the graph. The
%generation process starts with an empty graph and incrementally fills in the
%adjacency matrix by obeying the described constraints.

\paragraph{Realistic Social Network}
~\cite{Barrett:2009:GAL:1995456.1995598} focused on the construction of
realistic social networks using a combination of public and private data sets
and large-scale agent based techniques. The process works as follows: In the first step
it creates a synthetic population by integrating databases from commercial and
public sources. In the second step, a set of activity templates are determined. Each
synthetic individual is assigned a 24-hour activity sequence including
geolocations for each activity. To demonstrate the approach, the authors develop a synthetic population for the
United States that models every individual in the population. The synthetic
population is a set of geographically located people and households. Household
structure and demographics are derived from U.S. Census data. The activity
templates are  based on several thousand responses to an activity or time-use
survey. Demographic information for each person and location, a minute-by-minute
schedule of each person's activities, and the locations where these activities
take place is generated by a combination of simulation and data fusion
techniques. This information is captured by a dynamic social contact network. Similar methods for agent-based strategies have been reported in~\cite{Bernstein:2013:SAS:2499604.2499609}.

\paragraph{Linkage vs. Activity Graphs} \cite{Yao2011} distinguished between two
types of social network graphs -- the \emph{linkage graph}, where nodes stand
for the people in the social network and edges are their friendship links, and
the \emph{activity graph}, where nodes also stand for the people but edges stand
for their interactions. On the basis of the analysis of
Flickr\footnote{\url{https://www.flickr.com/}} social links and
Epinions\footnote{\url{http://www.epinions.com/}} network of user interactions,
the authors discover that they both exhibit power-law degree distribution, high
clustering coefficient (community structure), and small diameter; also regarding
the dynamic properties they both follow the densification law and have relatively stable
clustering coefficient over time. However, the authors do not observe diameter
shrinking in opinions activity graph and there is a difference in degree
correlation (how frequently nodes with similar degrees connect to each other).
Namely linkage graphs have positive degree correlation whereas activity graphs
show neutral degree correlation. With regards to the findings, the proposed generator focusses on linkage graphs
with positive degree correlation. For this purpose it extends the forest
fire spreading process algorithm~\cite{Leskovec:2005:GOT:1081870.1081893} with
link symmetry. It has two parameters -- the \emph{burning probability} $P_b$
which is in charge of the burning process, and the \emph{symmetry probability}
$P_s$ which indicates backward linking from old nodes to new ones. $P_b$
controls a BFS-based forward burning process. The fire burns increasingly
fiercely with $P_b$ approaching 1. Meanwhile, $P_s$ adds fuel to the fire as it
brings more links. It gives chances for big nodes to connect back to big nodes.


\paragraph{LinkBench} The LinkBench
benchmark~\cite{Armstrong:2013:LDB:2463676.2465296} has been designed to predict the
performance of a database when used for persistent storage of Facebook's
production data. The benchmark considers true Big Data and related problems with
sharding, replication etc. The social graph at Facebook comprises objects (nodes
with IDs, version, timestamp and data) and associations (directed edges, pairs
of node IDs, with visibility, timestamp and data). The size of the target graph
is the number of nodes. Graph edges are generated concurrently with graph nodes
during bulk loading. The node ID space is divided into chunks based on the ID of
the source node which  are processed in parallel. The edges of the graph are
generated in accordance with the results of analysing  real-world Facebook data
(such as outdegree distribution). A workload corresponding to 10 graph
operations (such as insert object, count the number of associations etc.) and
their respective characteristics over the real-world data is generated for the
synthetic data.

\paragraph{S3G2} The Scalable Structure-correlated Social Graph Generator
(S3G2)~\cite{Pham2013} is a general framework which  generates a directed
labeled graph, where the nodes are objects with property values, and their
structure is determined by the class a node belongs to. S3G2 does not aim at
generating near real-world data, but at generating synthetic graphs with a
correlated structure. It causes that the probability to choose a certain
property value (from a pre-defined dictionary), or the probability to connect
two nodes with an edge are influenced by existing data values. For example, it
is possible to have a correlated degree distribution, from which the degree of
each node is generated, correlated with the properties of node. Hence the generator
can ensure that, e.g., people with many friends in a social network will
typically post more pictures than people with few friends, i.e., the amount of
friend nodes influences the amount of posted comment and picture nodes. The data generation process starts with generating a number of nodes with
property values generated according to specified property value correlations and
then adding respective edges according to specified correlation dimensions. It
has multiple phases, each focusing on one correlation dimension. Each pass along
one correlation dimension is a Map phase in which data is generated, followed by
a Reduce phase that sorts the data along the correlation dimension in the next
pass. A heuristic observation that ``the probability that two nodes are
connected is typically skewed with respect to some similarity between the
nodes'' enables to focus only on sliding window of most probable candidates. The core idea of the framework is demonstrated using an example of a social
network (consisting of persons and social activities).  The dictionaries for
property values are inspired by DBpedia and provided with 20 property value
correlations. The edges are generated according to 3 correlation dimensions.


\paragraph{Cloning of Social Networks} Paper~\cite{Sukthankar-SocialInfo2014}
introduces a synthetic network generator designed for cloning social network
statistics of an existing dataset. The network starts with a small number of
nodes, and new nodes are added until the network reaches the required number. It
has two basic parameters: homophily and link density. A high \emph{homophily}
value indicates that links are more likely to be formed between nodes with the
same label; these labels can be viewed as being equivalent to community
membership.

Attribute Synthetic Generator (ASG) is a network generator for reproducing the
node feature distribution of standard networks and rewiring the network to
preferentially connect nodes that exhibit a high feature similarity. The network
is initialized with a group of three nodes, and new nodes and links are added to
the network based on link density, homophily, and feature similarity. As new
nodes are created, their labels are assigned based on the prior label
distribution. After the network has reached the same number of nodes as the
original social media dataset, each node initially receives a random attribute
assignment. Then a stochastic optimization process is used to move the initial
assignments closer to the target distribution extracted from social media
dataset using the Particle Swarm Optimization algorithm. The tuned attributes
are then used to add additional links to the network based on the feature
similarity parameter -- a source node is selected randomly and connected to the
most similar node. Multi-Link Generator (MLG) further  uses link co-occurrence statistics from the
original dataset to create a multiplex network. MLG uses the same network growth
process as ASG. Based on the link density parameter, either a new node is
generated with a label based on the label distribution of the target dataset or
a new link is created between two existing nodes.


\paragraph{LDBC SNB} The Social Network Benchmark
(SNB)~\cite{Erling:2015:LSN:2723372.2742786} provided by LDBC consists of three
distinct benchmarks on a common dataset corresponding to three different
workloads. SNB models a social network akin to Facebook. The dataset consists of
persons and a friendship network that connects them; whereas the majority of the
data is in the messages that these persons post in discussion trees on their
forums. The three query workloads involve: (1) SNB-Interactive, i.e., complex
read-only queries, that touch a significant amount of data, (2) SNB-BI which
accesses a large percentage of all entities in the dataset and groups these in
various dimensions, and (3) SNB-Algorithms, i.e., graph analysis algorithms,
including PageRank, Community Detection, Clustering and Breadth First Search.
The graph generator realizes power laws, uses skewed value distributions, and
introduces plausible correlations between property values and graph structures.
It is implemented on top of Hadoop to provide scalability.

%The generated data have become a part if several graph benchmarks, such as GraphBIG~\cite{Nai:2015:GUG:2807591.2807626}.

\paragraph{Towards More Realistic Data} \cite{Nettleton2016} argued that the main body of existing work lies in
topology generation which approximates the characteristics of a real social
network (such as a small graph diameter, small average path length, skew degree
distribution, and community structures), however,  this is usually done without any data. Hence, they
introduced a general stochastic modeling system which allows the users to
populate a graph topology with data. The approach has three steps: (1) topology
generation (using R-MAT) plus community identification using the Louvain
method~\cite{1742-5468-2008-10-P10008} or usage of a real-world topology from
SNAP\footnote{\url{https://snap.stanford.edu/data/}}, (2) data definition
following distribution profiles, attribute value definitions, using a
parameterizable set of data propagation rules and affinities, and (3) data
population.




\subsection{Graph Analytics}
\label{sec:generators_analytics}

\paragraph{HPC Scalable Graph Analysis Benchmark} The HPC Scalable Graph
Analysis Benchmark~\cite{HPCgraph,Bader:2005:DIH:2099301.2099360} represents an
application with multiple analysis techniques that access a single data
structure representing a weighted, directed graph. The benchmark is composed of
four separated operations (graph construction, classification of large vertex
sets, graph extraction with BFS, and graph analysis with betweenness centrality)
on a graph that follows a power-law distribution. The graph generator constructs a list of edge tuples containing vertex
identifiers (with the edge direction from the first one to the second one) and
weights that represent data assigned to the edges of the multigraph in the form
of positive integers with a uniform random distribution. The generator has the
following parameters: number of vertices, number of edges, and maximum weight of
an edge. The algorithm of the generator is based on R-MAT~\cite{DBLP:conf/sdm/ChakrabartiZF04}. Since the authors aim
to avoid data locality, in the final step the vertex numbers are randomly
permuted, and then edge tuples randomly shuffled. A related project from the same authors developed for the 9th DIMACS Shortest
Paths Challenge is GTgraph~\cite{GTgraph}. It involves three types of graphs:
input graph instances used in the DARPA HPCS SSCA\#2 graph theory benchmark
(version 1.0), Erd\"{o}s-R\'{e}nyi random graphs, and small-world graphs based
on the R-MAT model~\cite{DBLP:conf/sdm/ChakrabartiZF04}. 

Graphalytics~\cite{Iosup:2016:LGB:3007263.3007270} is an industrial-grade benchmark for graph analysis platforms. It involves 6 real-world datasets and 2 synthetic datasets generated that cover two commonly used type of graphs: social network graphs generated using LDBC SNB graph generator (see Section~\ref{sec:generators_socialnetworks}) and power-law graphs generated by Graph500 (see Section~\ref{sec:generators_general}). The benchmark workload consists of 6 deterministic algorithms: breadth-first search, PageRank, weakly connected components, community detection using label propagation, local clustering coefficient, and single-source shortest paths. The benchmark uses various metrics to measure the performance and throughput for systems under test such as \emph{upload time} the measures the required time to preprocess and convert the graph into a suitable format for a graph processing system and \emph{Makespan} the measures the total  execution time of a benchmarking workload algorithm. The benchmark also describes experiments for measuring the scalability of the systems under test. To facilitate the end user job of running the designed experimental workload, the benchmark provides  a performance evaluation framework, \texttt{Granula}\footnote{\url{https://github.com/atlarge-research/granula}}, that consists of three main components: the modeler, the archiver, and the visualizer.




\subsubsection{Graph Data Streaming}
\label{sec:generators_streaming}

Phuoc et al.~\cite{le2012linked} presented an evaluation framework for linked stream data processing engines. The framework uses a data generator for Stream Social network data Generator (\texttt{S2Gen}) that simulates what users continuously generate on their social network activities (e.g., posts) in addition to the  user metadata such as users' profile information, social network relationships, posts, photos and GPS information. The data generator of this framework provides the users the flexibility to control the characteristics of the generated data by tanning a range of parameters including the period in which the social activities are generated (generating period), the maximum number of posts/comments/photos for each user per week and the� correlation probabilities between the different objects (e.g., users) in the social network. 

Tommasini et al.~\cite{tommasini2017rsplab} introduced another framework for benchmarking RDF Stream Processing systems, \texttt{RSPLab}. The \texttt{Streamer} component of this framework is designed to publishe RDF streams from the various existing RDF benchmarks (e.g., BSBM, LUBM) (See Section~\ref{sec:generators_LinkedData}). In particular, the \texttt{Streamer}  component uses \texttt{TripleWave}, an open-source framework for publishing
and sharing RDF streams on the Web~\cite{mauri2016triplewave}.   \texttt{TripleWave}\footnote{\url{http://streamreasoning.github.io/TripleWave/}} acts as a mean for plugging in diverse Web data sources and for consuming streams in both push and pull mode.



% \subsection{Specific Types of Graphs}
% ...

% % Note: not in timeline or tables
% \paragraph{Heterogeneous Graphs} The Heterogeneous Graph Data Benchmark (GDB-H)~\cite{Gupta:2012:GLH:2741795.2741808} aims at \emph{heterogenous graph} data model, a mixed model graph structure that combines several existing generation techniques into a single benchmark. The idea is demonstrated using a drug discovery scenario whose schema involves 11 entity categories (e.g., genes, proteins, diseases, ...) and 3000 binary relationships (e.g., instanceOf, subclassOf, ...). The data is structured as a combination of $N$ overlapping named graphs $G_1, ... G_N$, where the overlap is accomplished by node sharing. A subset of the named graphs $G_1, ... G_k$ are hierarchical, i.e., they are structured as trees or DAGs. The remaining $N-k$ graphs are multigraphs which differ in terms of their network connectivity properties (some component graphs obey the power-law more strictly, some graphs have a larger skew in the distribution of edge labels, some graphs  are denser, some graphs may optionally have additional constraints regarding subgraph patterns). The user can specify the number of component graphs (8 to 100), the number of nodes (100,000 to 100,000,000), the number of node types (3 to 11), and the number of distinct edge labels (30 - 3000), and optionally also type ratios (the fraction of the component graphs having hierarchical, power-law, community or motif structure), node distribution (i.e. the relative size of graph components), edge density, overlapping etc.

% For the purpose of generating the heterogeneous graphs having heterogeneous, i.e. hierarchical, power-law, community-structured or purely random, structure the authors combine several existing approaches corresponding to the particular structural type~\cite{PhysRevLett.102.128701,doi:10.1080/10586458.2001.10504428}.


% \subsection{Object-Oriented and XML Databases (?)}

% see~\cite{Dominguez-Sal:2010:DDG:1946050.1946053}
