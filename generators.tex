\section{Graph Data Generators}
\label{sec:generators}

In this section, we discuss in more detail the different graph data generators using the classification introduced before. For each category we briefly describe the key features of each of the representative examples. The aim is to provide the readers with a detailed look at each of the tools in the context of its competitors from the same domain.


\subsection{General Graphs}
\label{sec:generators_general}

We start by focusing on approaches that have been designed for dealing with the generation of general graph
data that is not aimed at a particular application domain. Currently, there exists a
number of tools which involve a kind of general graph data generator, such as
gtools from projects nauty and Traces~\cite{gtools} or the Stanford
GraphBase~\cite{GraphBase}. We, however, focus on primarily
generating/benchmarking projects targeting the Big Data world.


\paragraph {Preferential Attachment} Barabasi and Albert~\cite{Barabasi99emergenceScaling} introduced a graph generation model that relies on two main mechanisms. The first mechanism is continuously expanding the graphs by adding new vertices. The second mechanism is to preferentially attach the new vertices   to the nodes/regions that are already well connected. So, in this approach, the generation of large graphs is governed by standard, robust self-organizing mechanisms that go beyond the characteristics of individual applications.

\paragraph {R-Mat} (\emph{R}ecursive \emph{Mat}rix) R-Mat is a procedural synthetic graph generator which is designed to generate power-law degree
distributions~\cite{DBLP:conf/sdm/ChakrabartiZF04}.
The generator is recursive and uses only a small number of parameters.
In principle, the strategy of this generator is to find simple mechanisms to generate graphs that match
the properties of the real graphs. In particular, the design goals of R-Mat is to generate graphs that match the degree distributions, imitate a community structure and have a small diameter. R-Mat can generate weighted, directed and bipartite graphs.

\paragraph{GraphGen} For the purpose of testing the scalability of an indexing
technique called FG-index~\cite{Cheng:2007:FTV:1247480.1247574} on the size of the
database of graphs, their average size and average density, the authors have
also implemented a synthetic generator called
GraphGen~\footnote{https://www.cse.ust.hk/graphgen/}. It is
based on the IBM synthetic data generation code for associations and sequential
patterns\footnote{From 1996, no longer available at
\url{http://www.almaden.ibm.com/cs/projects/iis/hdb/Projects/data
mining/mining.shtml}}. GraphGen creates a collection of labeled, undirected and
connected graphs which focuses on the performance evaluation of frequent subgraph mining
algorithms and graph query processing algorithms. The result is represented as a
list of graphs, each consisting of a list of nodes and a list of edges.




\paragraph{Graph 500 Benchmark} The Graph 500 Benchmark~\cite{Graph500} includes
a scalable data generator which produces weighted, undirected graph as a list of
edge tuples containing the label of start vertex and end vertex together with a
weight that represents data assigned to the edge. The space of vertex  labels is
the set of integers beginning with 0. The input values required to describe the
graph are (1) scale, i.e., the logarithm base two of the number of vertices, and
(2) edge factor, i.e., the ratio of the graph's edge count to its vertex count
(i.e., half the average degree of a vertex in the graph). The graph generator is
a Kronecker generator similar to R-MAT. The data must not exhibit any locality,
so in the final step the vertex labels and order of edges are randomly shuffled.
The covered operations currently involve BFS; however, the authors intend to
involve also two more types -- optimization (single source shortest path) and
edge-oriented (maximal independent set) -- and five graph-related business areas:
cybersecurity, medical informatics, data enrichment, social networks, and
symbolic networks.

\paragraph{BTER} BTER (Block Two-Level
Erd\"{o}s-R\'{e}ny)~\cite{kolda2014scalable} is a graph generator based on the
creation of multiple Erd\"{o}s-R\'{e}ny graphs with different connection
probabilities  of which they are connected randomly between them. As the main feature, BTER is able
to reproduce input degree distributions and average clustering
coefficient per degree values. The generator starts by grouping the vertices
by degree $d$, and forming groups of size $d+1$ of nodes with degree $d$. Then, these
groups are assigned an internal edge probability in order to match the observed
average clustering coefficient of the nodes of such degree. Based on this
probability, for each node, the excess degree (i.e, the degree that in
expectation will not be realized internally in the group) is computed and used to connect
nodes from different groups at random. The authors describe a highly scalable
MapReduce based implementation that is capable of generating large graphs (with
billions of nodes) in a reasonable amounts of time.

\paragraph{Darwini} Darwini~\cite{edunov2016darwini} is an extension of BTER
designed to run on Vertex Centric computing frameworks like
Pregel~\cite{malewicz2010pregel} or Apache Giraph~\cite{ching2015one}, with the
additional feature that it is more accurate when reproducing the clustering
coefficient of the input graph. Instead of just focusing on the average
clustering coefficient for each degree, Darwini is able to model the clustering
coefficient distribution per degree. It achieves this by grouping the vertices
of the graph into buckets by the expected number of closed triangles that they need
to close in order to attain the expected clustering coefficient, which is
sampled from the input distributions. Then, the vertices in each bucket are
connected randomly with a probability that would produce the expected 
desired number of triangles for such bucket. Then, as in BTER, the excess degree
is used to connect the different buckets. The authors report that Darwini is
able to generate graphs with billions and even trillions of edges.

%\paragraph{gtools} Projects nauty and Traces~\cite{gtools} are programs for
%computing automorphism groups of graphs and digraphs~\cite{McKay201494}; they
%can also produce a canonical label. In the respective package there is also
%suite of programs called gtools which involve generators for non-isomorphic
%graphs, bipartite graphs, digraphs, and multigraphs.





\subsection{Semantic Web}
\label{sec:generators_LinkedData}
With the dawn of the concept of Linked Data it is a natural development that there would emerge respective benchmarks involving both synthetic data and real-world data sets  sets with real-world characteristics. The used data sets correspond to RDF representation of relational-like data~\cite{Guo2005158,Bizer09theberlin}, social network-like data~\cite{Schmidt2010}, or specific and significantly more complex data structures such as biological data~\cite{Wu2014}. In this section, we provide an overview of benchmarking systems involving a kind of graph-based RDF data generator or data modifier. %Other types of systems or particular results can be found, e.g., at~\cite{RdfStoreBenchmarking}.

\iffalse
Considering the Big Data world, the Linked Data in general definitely belong to this group since we assume that the Linked (Open) Data Sets form a common Linked Open Data cloud\footnote{\url{http://lod-cloud.net/}}. On the other hand, the particular data sets can be relatively small.
\fi

\paragraph{LUBM} The use-case driven Lehigh University Benchmark (LUBM)\footnote{\url{http://swat.cse.lehigh.edu/projects/lubm/}} considers the university domain. The ontology defines 43 classes and 32 properties~\cite{Guo2005158}. In addition, the LUBM benchmark provides 14 test queries. In particular, the benchmark focuses on \emph{extensional} queries, i.e., queries which target the particular data instances of the ontology, as an opposite to \emph{intentional} queries, i.e., queries which target properties and classes of the ontology. The Univ-Bench Artificial  (UBA) data generator features repeatable and random  data generation (exploiting classical linear congruential generator, LCG, of numbers). In particular, the data which is produced by the generator are assigned zero-based indexes (i.e., the first university is named \emph{University0} and so on), thus they are reproducible at any time with the same indexes.  The generator naturally allows to specify a seed for random number generation, along with the desired number of universities, and the starting index of the universities.

An extension of LUBM, the Lehigh BibTeX Benchmark (LBBM)~\cite{Wang2005}, enables generating synthetic data for different ontologies. The data generation process is managed through two main phases: (1) the property-discovery phase, and (2) the data generation phase. LBBM provides a probabilistic model that can emulate the discovered properties of the data of a particular domain and generate synthetic data exhibiting similar properties. A Monte Carlo algorithm is employed to output synthetic data. The approach is demonstrated on the Lehigh University BibTeX ontology which consists of 28 classes along with 80 properties. The LUBM benchmark includes 12 test queries that were designed for the benchmark data. Another extension of LUBM, the University Ontology Benchmark (UOBM)\footnote{\url{https://www.cs.ox.ac.uk/isg/tools/UOBMGenerator/}}, focuses on two aspects: (1) usage of all constructs of OWL Lite and OWL DL~\cite{owl} and (2) lack of necessary links between the generated data which thus form isolated graphs~\cite{Ma:2006:TCO:2094613.2094629}. In the former case the original ontology is replaced by the two types of extended versions from which the user can choose. In the latter case cross-university and cross-department links are added to create a more complex graph.

\paragraph{IIMB} Ferrara et al.~\cite{Ferrara08OM} proposed the ISLab Instance Matching Benchmark (IIMB)\footnote{\url{http://www.ics.forth.gr/isl/BenchmarksTutorial/}} for the problem of instance matching. For any two objects $o_1$ and $o_2$ adhering to the same ontology or to different ontologies, instance matching is specified in the form of a function $Om(o_1, o_2) \rightarrow \{0, 1\}$,  where $o_1$ and $o_2$ are linked to the same real-world object (in which case the function maps to $1$) or $o_1$ and $o_2$ are representing different objects (in which case the function maps to $0$). It targets the domain of movie data which contains 15 named classes, along with 5 object properties and 13 datatype properties. The data are extracted from IMDb\footnote{\url{http://www.imdb.com/}}. The data generator corresponds to a data modifier which simulates differences between the data. In particular it involves data value differences (such as typographical errors or usage of different standard formats, e.g., for names), structural heterogeneity (represented by different levels of depth for properties, diverse aggregation criteria for properties, or missing values specification) and logical heterogeneity (such as instantiation on different subclasses of the same superclass or instantiation on disjoint classes).


\paragraph{BSBM} The Berlin SPARQL Benchmark (BSBM)\footnote{\url{http://wifo5-03.informatik.uni-mannheim.de/bizer/berlinsparqlbenchmark/}}, is centered around an e-commerce application domain with object types such as \emph{Customer}, \emph{Vendor}, \emph{Product} and \emph{Offer} in addition to the relationship among them~\cite{Bizer09theberlin}.
The benchmark provides a workload that has 12 queries with 2 types of query workloads (i.e., 2 sequences of the 12 queries) emulating the navigation pattern and search of a consumer seeking a product. The data generator is capable of producing arbitrarily scalable datasets by controlling the number of products ($n$) as a scale factor.  The scale factor also impacts other data characteristics, such as, e.g., the depth of type hierarchy of products, branching factor, the number of product features,  etc. BSBM can output two representations, i.e. an RDF representation along with a relational representation. Thus, BSBM also defines an SQL~\cite{sql} representation of the queries. This allows comparison of SPARQL~\cite{sparql} results  to be compared against the performance of traditional RDBMSs.


\paragraph{SP$^2$Bench} The SP$^2$Bench\footnote{\url{http://dbis.informatik.uni-freiburg.de/forschung/projekte/SP2B/}} is a language-specific benchmark~\cite{Schmidt2010} which is based on the DBLP dataset. %, so the types involve Person, Inproceedings, Article etc.
The generated datasets follow the key characteristics of the original DBLP dataset. In particular, the data mimics the correlations between entities. All random functions of the generator use a fixed seed that ensures that the data generation process is deterministic. SP$^2$Bench is accompanied by 12 queries covering the various types of operators such as RDF access paths in addition to typical RDF constructs.

%\paragraph{JustBench} ~\cite{Bail:2010:JFO:1940281.1940285} ...




\paragraph{DBPSB} DBpedia SPARQL Benchmark (DBPSB)\footnote{\url{http://aksw.org/Projects/DBPSB.html}} proposed at the University of Leipzig has been designed using workloads that have been issued by humans and applications over existing RDF data~\cite{Morsey2011,Morsey:2012:UBR:2900929.2901031}. In addition, the authors argue that benchmarks like LUBM, BSBM, or SP$^2$Bench resemble relational database benchmarks involving relational data structures with few and homogeneously structured classes, whereas, in reality, RDF datasets are increasingly heterogeneous. For example, DBpedia version 3.6 consists of 289,016 classes of which 275 classes are defined based on the DBpedia ontology. In addition, different data types and object references of the various types are
used in property values. Hence, they presented a universal SPARQL benchmark generation approach which uses a flexible data production mechanism that mimics the input data source. This dataset generation process begins using an input dataset; then multiple datasets with different sizes  are then generated by duplicating all the RDF triples with changing their namespaces.  For generating smaller datasets, an adequate selection of all triples is selected randomly or using a sampling mechanism over the various classes in the dataset. \iffalse The goal of the query analysis and clustering is to detect prototypical queries on the basis of their frequent usage and similarity.\fi The methodology is applied on the DBpedia SPARQL endpoint and a set of 25 SPARQL query templates is derived to cover the frequently used SPARQL features.

\paragraph{LODIB} The Linked Open Data Integration Benchmark (LODIB)\footnote{\url{http://lodib.wbsg.de/}} has been designed with the aim of reflecting the real-world heterogeneities that exist on the Web of Data in order to enable testing of Linked Data translation systems~\cite{DBLP:conf/www/RiveroSBR12}. It provides a catalogue of 15 data translation patterns (e.g., rename class, remove language tag etc.), each of which is a common data translation problem in the context of Linked Data. The benchmark provides a data generator that produces three different synthetic data sets that need to be translated
by the system under test into a single target vocabulary. They  reflect the pattern distribution in analyzed 84 data translation examples from the LOD Cloud. The data sets reflect the same e-commerce scenario used for BSBM.




\paragraph{Geographica} The Geographica benchmark\footnote{\url{http://geographica.di.uoa.gr/}} has been designed to target the area of geospatial data~\cite{DBLP:conf/semweb/GarbisKK13} and respective SPARQL extensions GeoSPARQL~\cite{battle2012enabling} and stSPARQL~\cite{koubarakis2010modeling}. The benchmark involves a real-world workload that uses openly available datasets that cover a range of geometry types (e.g., points, lines, polygons) and  a synthetic workload. In the former case there is a (1) a micro benchmark that evaluates primitive spatial functions (involving 29 queries) and (2) macro benchmark that tests the performance of RDF engines in various application scenarios such as  map exploring and search (consisting of 11 queries). In the latter case of a synthetic workload the generator produces synthetic datasets of different sizes that corresponds to an ontology based on OpenStreetMap  and instantiates query templates. \iffalse The spatial extent of the land ownership dataset constitutes a uniform grid of $n \times n$ hexagons, whereas the size of each dataset is given relatively to $n$.\fi The generated SPARQL query workload is corresponding to spatial selection and spatial joins by instantiating 2 query templates.


\paragraph{WatDiv} The Waterloo SPARQL Diversity Test Suite (WatDiv)\footnote{\url{http://dsg.uwaterloo.ca/watdiv/}} has been designed at the University of Waterloo. It implements stress testing tools that focus on addressing the observation that existing SPARQL benchmarks are not adequate for evaluating systems using a verity of queries and  workloads~\cite{Aluc:2014:DST:2717213.2717229}. The benchmark focuses on two types of query aspects -- structural and data-driven -- and performs a detailed analysis on existing SPARQL benchmarks (LUBM, BSBM, DBPSB, and SP$^2$Bench) using the two classes of query features. The structural features involve triple pattern count, join vertex count, join vertex degree, and join vertex count. The data-driven features involve result cardinality and several types of selectivity. The analysis of the four benchmarks reveals that they are not sufficiently diverse to evaluate the strengths and weaknesses of the various physical design alternatives that have been implemented by the different RDF systems. In particular, WatDiv, provides (1) a data generator which generates scalable datasets according to the WatDiv schema, (2) a query template generator which follows the WatDiv schema and produces a  set of query templates, (3) a query generator that uses the generated templates and instantiates them  with real RDF values from the dataset, and (4) a feature extractor which extracts the structural features of the generated data and workload. %For the study in the paper the authors generated 12,500 test queries from 125 query templates.

\paragraph{RBench} RBench~\cite{Qiao:2015:RAR:2723372.2746479} is an application-specific benchmark which receives any RDF dataset as an input and produces as an output a set of datasets, that have similar characteristics of the input dataset, using size scaling factor $s$ and (node) degree scaling factor $d$. These factors ensure that the original RDF graph $G$ and the synthetic graph $G'$ are similar and the number of edges and the average node degree of $G'$ are changed by $s$ and $d$ respectively. \iffalse A generated benchmark dataset is considered similar to the given dataset if their values for the dataset evaluation metrics and query evaluation times for different techniques are similar. Three evaluation metrics are utilized for this purpose: dataset coherence (i.e., a measure how uniformly predicates are distributed among the same type/class), relationship specialty (i.e., the number of occurrences of the same predicate associated with each resource), and literal diversity.\fi A query generation process has been implemented to produce 5 different types of queries (edge-based queries, node-based queries, path queries, star queries, subgraph queries) for any generated data. The benchmark project FEASIBLE~\cite{Saleem2015} is also an application-specific benchmark; however, contrary to RBench, it is designed to produce benchmarks from a set of queries (in particular from query logs) by relying on sample queries of a user-defined
size from the input set of queries.

In practice, one way for handling big RDF graphs is to process them using the
\emph{streaming} mode where the data stream could consist of the edges of the
graph. In this mode, the RDF processing algorithms can process the input
stream in the order it arrives while using only a limited amount of
memory~\cite{mcgregor2014graph}. The streaming mode has mainly  attracted the attention of the
RDF and Semantic Web community.

\paragraph{S2Gen}   Phuoc et al.~\cite{le2012linked} presented
an evaluation framework for linked stream data processing engines. The framework
uses a datasets generated with the Stream Social network data Generator
(S2Gen), which
simulates streams of user interactions (or events) in social networks
(e.g., posts) in addition to the  user metadata such as users' profile
information, social network relationships, posts, photos and GPS information.
The data generator of this framework provides the users the flexibility to
control the characteristics of the generated stream by tanning a range of
parameters, which includes the frequency at which interactions are generated,
limits such as the maximum number of messages per user
and week, and the correlation probabilities between the different objects (e.g.,
users) in the social network.

\paragraph{RSPLab} Tommasini et al.~\cite{tommasini2017rsplab} introduced
another framework for benchmarking RDF Stream Processing systems, RSPLab. The
Streamer component of this framework is designed to publish RDF streams from the
various existing RDF benchmarks (e.g., BSBM, LUBM) (see Section~\ref{sec:generators_LinkedData}).
In particular, the Streamer  component uses TripleWave\footnote{\url{http://streamreasoning.github.io/TripleWave/}}, an
open-source framework for publishing and sharing RDF streams on the
Web~\cite{mauri2016triplewave}.   TripleWave acts as a means for plugging-in
and combining streams from multiple Web data sources, in both push and pull mode.



\paragraph{LDBC}  The Linked Data Benchmark Council\footnote{\url{http://ldbcouncil.org/industry/organization/origins}} (LDBC)~\cite{Angles:2014:LDB:2627692.2627697} %is a result of a (closed) EU project that brought together a community of academic researchers and industry that
had the goal of developing an open source, yet industrial grade benchmarks for RDF and graph databases. \iffalse The following three benchmarks were developed and are currently maintained.\fi In the Semantic Web domain, it released the Semantic Publishing Benchmark (SPB)~\cite{spb} that has been inspired by the Media/Publishing industry (namely BBC\footnote{\url{http://www.bbc.com/}}). The application scenario of this benchmark simulates a media or a publishing organization that handles large amount of streaming content (e.g., news, articles). \iffalse This content is enriched with metadata that describes it and links it to reference knowledge -- taxonomies and databases that include relevant concepts, entities and factual information. The SPB data generator produces scalable in size synthetic large data. Synthetic data consists of a large number of annotations of media assets that refer entities found in reference datasets.\fi The data generator mimics three types of relations in the generated synthetic data: clustering of data, correlations of entities, and random tagging of entities. Two workloads are provided: (1) basic, involving an interactive query-mix querying the relations between entities in reference data, and (2) advanced,  focusing on interactive and analytical query-mixes. The LDBC has designed two other benchmarks: the Social Network Benchmark (SNB)~\cite{Erling:2015:LSN:2723372.2742786} for the social network domain  (see Section~\ref{sec:generators_socialnetworks}) and Graphalytics~\cite{Iosup:2016:LGB:3007263.3007270}   for the analytics domain.% (see Section~\ref{sec:generators_analytics}).



\paragraph{LinkGen} LinkGen is a synthetic linked data generator that has been designed to generate RDF datasets for a given vocabulary~\cite{10.1007/978-3-319-46547-0_12}. The generator is designed to receive a vocabulary as an input  and supports two statistical distributions for generating entities: Gaussian distribution and Zipf's power-law distribution. LinkGen can augment the generated data with inconsistent and noisy  data such as updating a given datatype property with two conflicting values or  adding triples with syntactic errors. \iffalse, adding wrong statements by assigning them with invalid domain and creating instances with no type information.\fi The generator also provides a feature to inter-link the generated objects with real ones given that the end-user provides entities from real datasets. The datasets can be generated in any of of two modes: on-disk and streaming.


\paragraph{Strengths and Weaknesses of Semantic Web Graph Generators.}  Graphs are intuitive and standard representation for the RDF model that form the basis for the Semantic Web community which has been very active on building several benchmarks, associated with graph generators that had various design principles. A comparison of 4 RDF benchmarks (namely TPC-H~\cite{TPC-H} data expressed in RDF, LUBM, BSBM, and SP$^2$Bench) and 6 real-wold data sets (such as, e.g.,  DBpedia, the Barton Libraries Dataset~\cite{barton-benchmark} or
WordNet~\cite{Miller:1995:WLD:219717.219748}) has been reported by~\cite{Duan:2011:AOC:1989323.1989340}. The authors focus mainly on the  \emph{structuredness} (\emph{coherence}) of each benchmark dataset claiming that a primitive metric (e.g., the number of triples or the average in/outdegree) quantifies only some target characteristics of each dataset. The degree of structuredness of a dataset $D$ with respect to a type $T$ is based on  the regularity of instance data in $D$ in conforming to type $T$. The type system is extracted from the data set by finding the RDF triples that have property  \texttt{http://www.w3.org/1999/02/22-rdf-syntax-ns\#type} and extract type $T$ from their object. Properties of $T$ are determined as the union of all the properties that the instances of type $T$ have. The structuredness is then expressed as a weighted sum of share of set properties of each type, whereas higher weights are assigned to types with more instances. The authors show that the structuredness of the chosen benchmarks is fixed, whereas real-world RDF datasets are belonging to the non-tested area of the spectrum. As a consequence, they introduce a new benchmark that receives as input any dataset associated with a required level of structuredness and size (smaller than the size of the original data), and exploits the input documents as a seed to produce a subset of the original data with the target structuredness and size. In addition, they show that structuredness and size mutually influence each other. With the recent increasing momentum of streaming data, the Semantic Web community started to consider the issues and challenges of RDF streaming data. However, there is still a lot of open challenges that needs to tackled in this direction such as covering different real-world application scenarios. 

\subsection{Graph Databases}
\label{sec:generators_GraphDatabases}

Currently there exists a number  of papers which compare the efficiency of graph databases with regards to distinct use cases, such as  the community detection problem~\cite{Beis2015}, social tagging systems~\cite{Giatsoglou2011}, graph traversal~\cite{Ciglan:2012:BTO:2374486.2375242}, graph pattern matching~\cite{Pobiedina2014}, data provenance~\cite{Vicknair:2010:CGD:1900008.1900067}, or even several distinct use cases~\cite{Grossniklaus2013Towar-24253}. However, the number of graph data generators and benchmarks that have been designed specifically for graph databases is relatively small. Either a general graph generator is used for benchmarking graph databases, such as, e.g., the HPC Scalable Graph Analysis Benchmark~\cite{Dominguez-Sal:2010:SGD:1927585.1927590} or the graph DBMS benchmarking tools are designed in a more general scope. Hence it is questionable whether a benchmark  that is targeted specifically for graph databases is necessary. \cite{Dominguez-Sal:2010:DDG:1946050.1946053} discussed this question and related topics. On the basis of a review of applications of graph databases (namely, social network analysis,  genetic interactions, recommendation systems, and travel planning and routing), the authors analyzed and discussed the features of the graphs for these types of applications and how they could affect the benchmarking process, different types of operations used in these applications and the characteristics of the evaluation setup of the benchmark. In this section, we focus on graph data generators and benchmarks that have been primarily targeting graph DBMSs.


\paragraph{XGDBench} XGDBench~\cite{Dayarathna:2014:GDB:2676904.2676939} is an extensible  benchmarking platform for graph databases used in cloud-based systems. Its intent is to automate
the process of graph database benchmarking in the cloud by focusing on the domain social networking services. It extends the Yahoo! Cloud Multiplicative Attribute (MAG) Graph Serving Benchmark (YCSB)~\cite{Cooper:2010:BCS:1807128.1807152} and provides a set of standard workloads representing various performance issues. In particular, the workload of XGDBench involves basic operations such as read / insert / update / delete an attribute, loading of the list of neighbors and BFS traversal. Using the generators, 7 workloads are created, such as update heavy, read mostly, short range scan, traverse heavy etc.
The data model of XGDBench is a simplified version of the Multiplicative Attribute Graph (MAG)~\cite{Kim2010} model, a synthetic graph model which models the interactions between  node attributes and  graph structure.
The generated graphs are thus in MAG format, with power-law degree distribution closely simulating real-world social networks.
The simplified MAG algorithm accepts the required number of nodes, the number of attributes per each node, a threshold value for random attribute initialization, an edge affinity threshold determining existence of an edge between two nodes, and an affinity matrix. \iffalse It has been proven that MAG generates graphs with both analytically tractable and statistically interesting properties.\fi 
Large graphs can be generated on multi-core systems by a multi-threaded version of the  generator.


\paragraph{gMark}  gMark~\cite{gMark} is a schema-driven and domain-agnostic generator of both graph instances and graph query workloads. It can generate instances under the form of N-triples and queries in various concrete query languages, including OpenCypher\footnote{\url{https://neo4j.com/developer/cypher-query-language/}}, recursive SQL, SPARQL and LogicQL. In gMark, it is possible to specify a \emph{graph configuration} involving the admitted edge predicates and node labels occurring in the graph instance along with additional parameters such as degree distribution, occurrence constraints, etc. The \emph{Query workload configuration} describes parameters of the query workload to be generated, by including the number of queries, arity, shape and selectivity of the queries.
The problem of deciding whether there exists a graph that satisfies a defined graph specification $G$ is NP-complete. The same applies to the problem of deciding
whether there exists a query workload compliant with a given query workload configuration $Q$. In view of this, gMark adopts a best effort approach in which the
parameters specified in the configuration files are attained in a relaxed fashion in order to achieve linear running time whenever possible.

%The authors prove that deciding whether there exists a graph satisfying a given graph configuration $G$ is NP-complete. And, similarly, deciding
%whether there exists a query workload satisfying a given query workload configuration $Q$ is also NP-complete. Hence, gMark generating is based on a heuristic strategy: it tries to achieve the exact values of the given parameters, however, in order to obtain linear running time it may decide to relax some. gMark generates graphs under the form of N-triples and query workloads in four concrete syntaxes, including Cypher\footnote{\url{https://neo4j.com/developer/cypher-query-language/}}, SPARQL, SQL and LogicQL.

\paragraph{GraphGen}  GraphAware GraphGen\footnote{\url{http://graphgen.graphaware.com/}} is a graph generation engine based on Neo4j's\footnote{\url{https://neo4j.com/}} query language Cypher~\cite{GraphGen}.  It creates nodes and relationships based on a schema definition expressed in Cypher, and it can also generate property values on both
nodes and edges. As such, GraphGen is a precursor of property graphs generators. The resulting graph can be exported to several formats (namely GraphJson\footnote{\url{https://github.com/GraphAlchemist/GraphJSON/wiki/GraphJSON}} and CypherQueries) or loaded directly to a DBMS. However, it is very likely that it is not maintained anymore due to the lack of available updates.


\subsection{Social Networks}
\label{sec:generators_socialnetworks}

On-line social networks, like Facebook, Twitter, or LinkedIn, have become a
phenomenon used by billions of people every day and thus providing extremely
useful information for various domains. However, an analysis of such type of
graph has to cope with two problems: (1) availability of the data and (2)
privacy of the data. Hence, data generators which provide realistic synthetic
social network graphs are in a great demand. 

In general, analyses of social networks identify their various specific
features~\cite{Chakrabarti:2006:GML:1132952.1132954}. For example, a social
graph has high \emph{clustering coefficient}, i.e. the degree of transitivity of
a graph. Or, its diameter, i.e. the minimum number of hops in which some
fraction (e.g., 90\%) of all connected pairs of nodes can reach each other, is
usually low due to weak ties joining faraway cliques. 

Another important aspect
of social networks is the community effect. A detailed study of structure of
communities in 70 real-world networks is provided, e.g., in
paper~\cite{Leskovec:2008:SPC:1367497.1367591}. Authors of
paper~\cite{Prat-Perez:2014:CSS:2621934.2621942} analyze the structure of
communities (clustering coefficient, triangle participation ratio, bridges,
diameter, conductance and size) in both real-world graphs and outputs of existing graph
generators LFR~\cite{PhysRevE.78.046110} and the
LDBC-SNB~\cite{Erling:2015:LSN:2723372.2742786}. They discover that communities found in different graphs follow quite similar distributions and that communities in a single graph have diverse nature and are difficult to fit with a single model.

The existing social network generators try to reproduce different aspects of the
generated network. They can be categorized into statistical and agent-based.
\emph{Statistical
approaches}~\cite{PhysRevE.78.046110,Yao2011,Armstrong:2013:LDB:2463676.2465296,Pham2013,Sukthankar-SocialInfo2014,Erling:2015:LSN:2723372.2742786,Nettleton2016}
focus on reproducing aspects of the network. In \emph{agent-based
approaches}~\cite{Barrett:2009:GAL:1995456.1995598,Bernstein:2013:SAS:2499604.2499609}
the networks are constructed by directly simulating the agents' social choices.

%\paragraph{LFR} Lancichinetti, Fortnato and Radicchi (hence
%LFR)~\cite{PhysRevE.78.046110} develop a class of benchmark graphs whose nodes
%participate in internal community structures. The benchmark models directed and
%weighted real-world networks (e.g., social networks) containing overlapping
%communities of different sizes. The algorithm assumes that both the degree and
%the community size distributions are power laws. Each node shares a fraction $(1
%- \mu)$ of its links with the other nodes of its community and a fraction $\mu$
%with the other nodes of the network, where $\mu$ is called \emph{mixing
%parameter}. The sizes of the communities are taken from a power law distribution
%such that the sum of all sizes equals the number of nodes of the graph. The
%generation process starts with an empty graph and incrementally fills in the
%adjacency matrix by obeying the described constraints.

\paragraph{Realistic Social Network}
~\cite{Barrett:2009:GAL:1995456.1995598} focused on construction of
realistic social networks using a combination of public and private data sets
and large-scale agent based techniques. The process works as follows: In the first step
it creates a synthetic population by integrating databases from commercial and
public sources. In the second step, a set of activity templates are determined. Each
synthetic individual is assigned a 24-hour activity sequence including
geolocations for each activity. To demonstrate the approach, the authors develop a synthetic population for the
United States that models every individual in the population. The synthetic
population is a set of geographically located people and households. Household
structure and demographics are derived from U.S. Census data. The activity
templates are  based on several thousand responses to an activity or time-use
survey. Demographic information for each person and location, a minute-by-minute
schedule of each person's activities, and the locations where these activities
take place is generated by a combination of simulation and data fusion
techniques. This information is captured by a dynamic social contact network. Similar methods for agent-based strategies have been reported in~\cite{Bernstein:2013:SAS:2499604.2499609}.

\paragraph{Linkage vs. Activity Graphs} \cite{Yao2011} distinguished between two
types of social network graphs -- the \emph{linkage graph}, where nodes stand
for the people in the social network and edges are their friendship links, and
the \emph{activity graph}, where nodes also stand for the people but edges stand
for their interactions. On the basis of analysis of
Flickr\footnote{\url{https://www.flickr.com/}} social links and
Epinions\footnote{\url{http://www.epinions.com/}} network of user interactions,
the authors discover that they both exhibit power-law degree distribution, high
clustering coefficient (community structure), and small diameter; also regarding
the dynamic properties they both follow densification law and relatively stable
clustering coefficient over time. However, the authors do not observe diameter
shrinking in Epinions activity graph and there is a difference in degree
correlation (how frequently nodes with similar degrees connect to each other).
Namely linkage graphs have positive degree correlation whereas activity graphs
show neutral degree correlation. With regards to the findings, the proposed generator focusses on linkage graphs
with positive degree correlation. For this purpose it extends the forest
fire spreading process algorithm~\cite{Leskovec:2005:GOT:1081870.1081893} with
link symmetry. It has two parameters -- the \emph{burning probability} $P_b$
which is in charge of the burning process, and the \emph{symmetry probability}
$P_s$ which indicates backward linking from old nodes to new ones. $P_b$
controls a BFS-based forward burning process. The fire burns increasingly
fiercely with $P_b$ approaching 1. Meanwhile, $P_s$ adds fuel to the fire as it
brings more links. It gives chances for big nodes to connect back to big nodes.


\paragraph{LinkBench} The LinkBench
benchmark~\cite{Armstrong:2013:LDB:2463676.2465296} has been designed to predict the
performance of a database when used for persistent storage of Facebook's
production data. The benchmark considers true Big Data and related problems with
sharding, replication etc. The social graph at Facebook comprises objects (nodes
with IDs, version, timestamp and data) and associations (directed edges, pairs
of node IDs, with visibility, timestamp and data). The size of the target graph
is the number of nodes. Graph edges are generated concurrently with graph nodes
during bulk loading. The node ID space is divided into chunks based on the ID of
the source node which  are processed in parallel. The edges of the graph are
generated in accordance with the results of analysis of real-world Facebook data
(such as outdegree distribution). A workload corresponding to 10 graph
operations (such as insert object, count the number of associations etc.) and
their respective characteristics over the real-world data is generated for the
synthetic data.

\paragraph{S3G2} The Scalable Structure-correlated Social Graph Generator
(S3G2)~\cite{Pham2013} is a general framework which  generates a directed
labeled graph, where the nodes are objects with property values, and their
structure is determined by the class a node belongs to. S3G2 does not aim at
generating near real-world data, but at generating synthetic graphs with a
correlated structure. It causes that the probability to choose a certain
property value (from a pre-defined dictionary), or the probability to connect
two nodes with an edge are influenced by existing data values. For example, it
is possible to have a correlated degree distribution, from which the degree of
each node is generated, correlated with properties of node. Hence the generator
can ensure that, e.g., people with many friends in a social network will
typically post more pictures than people with few friends, i.e., the amount of
friend nodes influences the amount of posted comment and picture nodes. The data generation process starts with generating a number of nodes with
property values generated according to specified property value correlations and
then adding respective edges according to specified correlation dimensions. It
has multiple phases, each focusing on one correlation dimension. Each pass along
one correlation dimension is a Map phase in which data is generated, followed by
a Reduce phase that sorts the data along the correlation dimension in the next
pass. A heuristic observation that ``the probability that two nodes are
connected is typically skewed with respect to some similarity between the
nodes'' enables to focus only on sliding window of most probable candidates. The core idea of the framework is demonstrated using an example of a social
network (consisting of persons and social activities).  The dictionaries for
property values are inspired by DBpedia and provided with 20 property value
correlations. The edges are generated according to 3 correlation dimensions.


\paragraph{Cloning of Social Networks} Paper~\cite{Sukthankar-SocialInfo2014}
introduces a synthetic network generator designed for cloning social network
statistics of an existing dataset. The network starts with a small number of
nodes, and new nodes are added until the network reaches the required number. It
has two basic parameters: homophily and link density. A high \emph{homophily}
value indicates that links are more likely to be formed between nodes with the
same label; these labels can be viewed as being equivalent to community
membership.

Attribute Synthetic Generator (ASG) is a network generator for reproducing the
node feature distribution of standard networks and rewiring the network to
preferentially connect nodes that exhibit a high feature similarity. The network
is initialized with a group of three nodes, and new nodes and links are added to
the network based on link density, homophily, and feature similarity. As new
nodes are created, their labels are assigned based on the prior label
distribution. After the network has reached the same number of nodes as the
original social media dataset, each node initially receives a random attribute
assignment. Then a stochastic optimization process is used to move the initial
assignments closer to the target distribution extracted from social media
dataset using the Particle Swarm Optimization algorithm. The tuned attributes
are then used to add additional links to the network based on the feature
similarity parameter -- a source node is selected randomly and connected to the
most similar node. Multi-Link Generator (MLG) further  uses link co-occurrence statistics from the
original dataset to create a multiplex network. MLG uses the same network growth
process as ASG. Based on the link density parameter, either a new node is
generated with a label based on the label distribution of the target dataset or
a new link is created between two existing nodes.


\paragraph{LDBC SNB} The Social Network Benchmark
(SNB)~\cite{Erling:2015:LSN:2723372.2742786} provided by LDBC consists of three
distinct benchmarks on a common dataset corresponding to three different
workloads. SNB models a social network akin to Facebook. The dataset consists of
persons and a friendship network that connects them; whereas the majority of the
data is in the messages that these persons post in discussion trees on their
forums. The three query workloads involve: (1) SNB-Interactive, i.e., complex
read-only queries, that touch a significant amount of data, (2) SNB-BI which
accesses a large percentage of all entities in the dataset and groups these in
various dimensions, and (3) SNB-Algorithms, i.e., graph analysis algorithms,
including PageRank, Community Detection, Clustering and Breadth First Search.
The graph generator realizes power laws, uses skewed value distributions, and
introduces plausible correlations between property values and graph structures.
It is implemented on top of Hadoop to provide scalability.

%The generated data have become a part if several graph benchmarks, such as GraphBIG~\cite{Nai:2015:GUG:2807591.2807626}.

\paragraph{Towards More Realistic Data} \cite{Nettleton2016} argued that the main body of existing work lies in
topology generation which approximates the characteristics of a real social
network (such as a small graph diameter, small average path length, skew degree
distribution, and community structures), however without data. Hence, they
introduced a general stochastic modeling system which allows the users to
populate a graph topology with data. The approach has three steps: (1) topology
generation (using R-MAT) plus community identification using the Louvain
method~\cite{1742-5468-2008-10-P10008} or usage of a real-world topology from
SNAP\footnote{\url{https://snap.stanford.edu/data/}}, (2) data definition
following distribution profiles, attribute value definitions, using a
parameterizable set of data propagation rules and affinities, and (3) data
population.




\subsection{Graph Analytics}
\label{sec:generators_analytics}

\paragraph{HPC Scalable Graph Analysis Benchmark} The HPC Scalable Graph
Analysis Benchmark~\cite{HPCgraph,Bader:2005:DIH:2099301.2099360} represents an
application with multiple analysis techniques that access a single data
structure representing a weighted, directed graph. The benchmark is composed of
four separated operations (graph construction, classification of large vertex
sets, graph extraction with BFS, and graph analysis with betweenness centrality)
on a graph that follows a power-law distribution. The graph generator constructs a list of edge tuples containing vertex
identifiers (with the edge direction from the first one to the second one) and
weights that represent data assigned to the edges of the multigraph in the form
of positive integers with a uniform random distribution. The generator has the
following parameters: number of vertices, number of edges, and maximum weight of
an edge. The algorithm of the generator is based on R-MAT~\cite{DBLP:conf/sdm/ChakrabartiZF04}. Since the authors aim
to avoid data locality, in the final step the vertex numbers are randomly
permuted, and then edge tuples randomly shuffled. A related project from the same authors developed for the 9th DIMACS Shortest
Paths Challenge is GTgraph~\cite{GTgraph}. It involves three types of graphs:
input graph instances used in the DARPA HPCS SSCA\#2 graph theory benchmark
(version 1.0), Erd\"{o}s-R\'{e}nyi random graphs, and small-world graphs based
on the R-MAT model~\cite{DBLP:conf/sdm/ChakrabartiZF04}.

\subsection{Others}
\label{sec:generators_others}
\subsection{Community Detection}
\label{sec:generators_community_detection}

Community detection is one of the many graph analytics algorithms typically used
on domains such as social networks or bioinformatics. Communities are groups of nodes that are highly connected among them, while being scarcely connected to
the rest of the graph. Such communities emerge from the fact that real graphs
are not random, but follow real-world dynamics that make similar entities to
have a larger probability to be connected. As a consequence, detected
communities are used to reveal vertices with similar characteristics, for
instance to discover functionally equivalent proteins in protein-protein
interaction networks, or persons with similar interests in social networks. Such
applications have made community detection a hot topic during the last fifteen
years with tens of developed algorithms and detection
strategies~\cite{doi:10.1002/wics.1403,Kim:2015:CDM:2854006.2854013}. For
comparing the quality of the different proposed techniques, one needs graphs
with \emph{reference communities}, that is, communities known beforehand. Since
it is very difficult to have large real graphs with reference communities
(mainly because these would require a manual labeling), graphs for benchmarking
community detection algorithms are typically generated synthetically.

\paragraph{Danon et al.} The first attempts to compare community detection algorithms using synthetic
graphs proposed the use of random graphs composed by several Erd\"{o}s-R\'{e}nyi
subgraphs, connected more internally than externally~\cite{danon2005comparing}.
Each of these subgraphs has the same size and the same internal/external density
of edges. However, such graphs miss the realism observed in real graphs, where
communities are of different sizes and densities, thus several proposals exist
to overcome such an issue.

\paragraph{LFR} Lancichinetti, Fortunato
and Radicchi (hence LFR)~\cite{PhysRevE.78.046110} propose a class of benchmark
graphs for community detection where communities are of diverse sizes and
densities. The generated communities follow a power-law distribution whose parameters can be configured. The degree of the
nodes is also sampled from a power-law distribution. Additionally, the generator
introduces the concept of the ``mixing factor'', which consists of the percentage of
edges in the graph connecting nodes that belong to different communities. Such parameter
allows  the degree of modularity of the generated graph  to be tuned, thus
testing the robustness of the algorithms under different conditions. The
generation process is implemented as an optimization process starting with an empty graph and 
progressively filling it with nodes and edges guided by the specified constraints.

\paragraph{LFR-Overlapping} Lancichinetti, Fortunato and
Radicchi~\cite{PhysRevE.80.016118} extended LFR to support the notion of
directed graphs and overlapping communities. Overlapping communities extend the
notion of communities by allowing the sharing of vertices, thus a vertex can
belong to more than one community. This extended generator allows controlling
the same parameters of LFR, as well as the amount of overlap of
the generated communities.

\paragraph{Strengths and Weaknesses of Community Detection Generators}
Besides synthetic graph generators, Yang and Leskovec~\cite{yang2015defining}
proposed the use of real-world graphs with explicit group annotations (e.g.,
forums in a social network, categories of products, etc.) to infer what they
call ``meta-communities'', and use them to evaluate overlapping community
detection algorithms. However, a recent study from Hric, Darst and
Fortunato~\cite{hric2014community} reveal a loose correspondence between
communities (the authors refer to them as structural communities) and
meta-communities.  This result reveals that  algorithms working for structural
communities do not work well for finding meta-communities and vice versa,
suggesting significantly different underlying characteristics
between the two types of communities, which are yet to be
identified.

In this regard and to the best of our knowledge, there does not exist a set of
generators to generate graphs with meta-communities for community detection
algorithm benchmarking. The closest one is the LDBC-SNB data
generator~\cite{Erling:2015:LSN:2723372.2742786} which has been provided by the
generation of groups of users in the social network. Even though the generation
process does not specifically enforce the generation of groups
(meta-communities) for benchmarking community detection algorithms, the study
from Prat-Pe\'rez and Dominguez-Sal reveals that these groups are more similar
to the real meta-communities than those structural communities generated by the
LFR benchmark~\cite{Prat-Perez:2014:CSS:2621934.2621942}.

The differences observed between structural and meta-communities reveal the need
of more accurate community definitions tight more specifically to the domain or
use case. Current community detection algorithms and graph generators for
community detection are stuck to the traditional (and vague) definition of
community, assuming that there exists a single algorithm that would fit all the
use cases. Thus, future work requires the study of domain-specific community
characteristics that can be used to generate graphs with a community structure
that accurately resembles that of specific use cases, and thus revealing which
are the best algorithms for each particular scenario.


\subsection{Graph Data Streaming}
\label{sec:generators_streaming}

...








% \subsection{Specific Types of Graphs}
% ...

% % Note: not in timeline or tables
% \paragraph{Heterogeneous Graphs} The Heterogeneous Graph Data Benchmark (GDB-H)~\cite{Gupta:2012:GLH:2741795.2741808} aims at \emph{heterogenous graph} data model, a mixed model graph structure that combines several existing generation techniques into a single benchmark. The idea is demonstrated using a drug discovery scenario whose schema involves 11 entity categories (e.g., genes, proteins, diseases, ...) and 3000 binary relationships (e.g., instanceOf, subclassOf, ...). The data is structured as a combination of $N$ overlapping named graphs $G_1, ... G_N$, where the overlap is accomplished by node sharing. A subset of the named graphs $G_1, ... G_k$ are hierarchical, i.e., they are structured as trees or DAGs. The remaining $N-k$ graphs are multigraphs which differ in terms of their network connectivity properties (some component graphs obey the power-law more strictly, some graphs have a larger skew in the distribution of edge labels, some graphs  are denser, some graphs may optionally have additional constraints regarding subgraph patterns). The user can specify the number of component graphs (8 to 100), the number of nodes (100,000 to 100,000,000), the number of node types (3 to 11), and the number of distinct edge labels (30 - 3000), and optionally also type ratios (the fraction of the component graphs having hierarchical, power-law, community or motif structure), node distribution (i.e. the relative size of graph components), edge density, overlapping etc.

% For the purpose of generating the heterogeneous graphs having heterogeneous, i.e. hierarchical, power-law, community-structured or purely random, structure the authors combine several existing approaches corresponding to the particular structural type~\cite{PhysRevLett.102.128701,doi:10.1080/10586458.2001.10504428}.


% \subsection{Object-Oriented and XML Databases (?)}

% see~\cite{Dominguez-Sal:2010:DDG:1946050.1946053}
