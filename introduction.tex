\section{Introduction}
\label{sec:intro}

%In recent years the term Big Data has become a phenomenon that breaks down borders of many technologies and approaches that have so far been acknowledged as mature and robust for any conceivable application. Gartner Inc.\footnote{\url{http://www.gartner.com/}} describes Big Data as ``\emph{high \textbf{v}olume, high \textbf{v}elocity, and/or high \textbf{v}ariety information assets that require new forms of processing to enable enhanced decision making, insight discovery and process optimization}''.

%\TODO{The abundance of interconnected data has fueled the design and implementation of graph generators reproducing real-world linking properties, or gauging the effectiveness of graph algorithms, techniques and applications manipulating these data.}

One of the most popular data structures in the world of computer science are graphs. They enable to represent complex information in a simple and yet powerful way supported by a strong mathematical background. With the dawn of the new millennium which brought numerous novel technologies and applications, the graphs have become even more popular. There has appeared a number of use cases\footnote{\url{https://neo4j.com/use-cases/}} (such as fraud detection, recommendation engines, social networks etc.) where graphs represent the linked information. 

In general, we can distinguish two types of graph data sets: (1) a single large graph (possibly with several components), such as social networks  or Linked Data graphs, and (2) a large set of small graphs, such as chemical compounds\footnote{\url{https://pubchem.ncbi.nlm.nih.gov/}} or linguistics syntax trees\footnote{\url{https://catalog.ldc.upenn.edu/ldc99t42}}. Naturally, the algorithms used in theses two classes differ a lot~\cite{DBLP:books/igi/Sakr2011}. In the former case we can search, e.g., for communities and their features or shortest paths, while in the latter case we usually query for supergraphs, subgraphs or graphs similar to a given graph pattern. Also, as in the other fields, quite often the respective real-world data is not publicly available (or simply does not exist when a particular method for manipulation with the data is proposed). As a consequence, also in the world of graphs there has appeared a number of graph data generators which enable to  reproduce the respective real-world linking properties and gauge the effectiveness of a particular algorithm.

In this survey we ... \TODO{We consider graph generation targeting multiple subfields, such as graph databases, graph data mining, graph streaming and machine learning communities, alongside community detection, social networks and IoT communities}

... \TODO{Despite the disparate requirements of graph generators throughout these communities, we analyze them under a common umbrella, reaching out the functionalities, the practical usage, and the supported operations of graph generators.}

... \TODO{We argue that this classification is serving the need of providing data scientists, researchers and practitioners with the right data generator at hand for their work.}

... \TODO{This survey provides a comprehensive overview of the state-of-the-art graph generators by focusing on those that are pertinent and suitable for data science tasks. Finally, we discuss open challenges and missing features of graph generators in view of the evolution of data science.}



\paragraph*{Contributions} The key contributions of this survey are as follows:
\begin{itemize}
  \item ...
\end{itemize}

...

\paragraph*{Differences with prior surveys}

To the best of our knowledge, our work is the first surveying the broad landscape of graph data generators spanning different Big Data applications and
targeting diverse computer science subfields. In particular, we cover graph database
generators, graph processing generators, social networks and community detection
generators, Semantic Web data generators, graph
analytics generators, IoT, Telecommunication and graph streaming generators,
and Machine Learning and graph mining generators. A comprehensive study of
graph generators is missing for many of the specific subfields mentioned
above.

A limited subset of graph database
generators, parallel and distributed graph processing generators, along
with a few of the Semantic Web data generators presented in our survey have
been discussed in a related book chapter
\cite{BFHI18} while cross-comparing them with respect to input, output,
supported workload, data model and query language, along with the
distinguished chokepoints. However, the provided classification is
inherently data-driven and not meant to serve the purpose of letting any
researcher or practitioner interested in Big Data be able to make a guided
choice of the desired
generator based on its functional and goal-driven features (such as the application domain,
the supported operations and the key configuration options).
Moreover, our work is much broader and targets graph generators of several
diversified communities, not limiting its scope to few generators of the database
and graph processing communities.

Graph generators matching graph patterns used in data mining have been
studied in \cite{Chakrabarti:2006:GML:1132952.1132954},
focusing on mostly occurring patterns, such as power laws, size of graph diameters
and community structure. The considered graph generators are compared in
terms of graph type, degree distributions, exponentiality, diameter and
community effects. We refer the reader to this survey for taxonomies
involving these properties, whereas we provide here a functionality-driven
taxonomy across all the categories of graph generators that we consider.
We also point out that this survey is outdated as it does not consider the
graph mining generators that fervently appeared in the last decade.

\TODO{Check with Arnau}

Aggarwal and Subbian \cite{AggarwalS14} have surveyed evolution analysis in
graphs, by primarily focusing on data mining maintenance methods and on analytical
quantification and explanation of the changes of the underlying networks.
A brief discussion on evolutionary
network data generators is carried out in
the paper. The data generation of evolutionary networks based on
Densification Power Law (DPL) and shrinking diameters \cite{LeskovecKF05} and community-guided
attachment properties \cite{LeskovecKF05} is then considered, along with tackling Kronecker
recursion with recursive tensor multiplication \cite{AkogluMF08}.

\TODO{Better highlight the differences when Sherif's part is in.}

%\TODO{Outreach other surveys in other communities.}


\paragraph*{Outline} The rest of the text is structured as follows: ...
% In Section~\ref{sec:preliminaries} we provide a brief introduction to graph theory and related terms.
% In Section~\ref{sec:classification} we classify the current Big graph Data applications.
% In Section~\ref{sec:generators} we introduce the existing graph data generators in the context of the proposed categories and their main features.
% Section~\ref{sec:comparison} provides comparison of the described generators.
% In Section~\ref{sec:challenges} we discuss challenges and open problems. And in Section~\ref{sec:conclusion} we conclude.


