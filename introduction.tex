\section{Introduction}
\label{sec:intro}

In recent years the term Big Data has become a phenomenon that breaks down borders of many technologies and approaches that have so far been acknowledged as mature and robust for any conceivable application. Gartner Inc.\footnote{\url{http://www.gartner.com/}} describes Big Data as ``\emph{high \textbf{v}olume, high \textbf{v}elocity, and/or high \textbf{v}ariety information assets that require new forms of processing to enable enhanced decision making, insight discovery and process optimization}''.

...

\TODO{there are two types of graph big data sets - large graphs and large sets of small graphs; we focus on the former one}

\paragraph*{Contributions} In this survey we will focus on graph data generators in the context of Big Data and related applications of graph algorithms. The key contributions are as follows:

\begin{itemize}
  \item ...
\end{itemize}

...

\paragraph*{Outline} The rest of the text is structured as follows: ...
% In Section~\ref{sec:preliminaries} we provide a brief introduction to graph theory and related terms.
% In Section~\ref{sec:classification} we classify the current Big graph Data applications.
% In Section~\ref{sec:generators} we introduce the existing graph data generators in the context of the proposed categories and their main features.
% Section~\ref{sec:comparison} provides comparison of the described generators.
% In Section~\ref{sec:challenges} we discuss challenges and open problems. And in Section~\ref{sec:conclusion} we conclude.

