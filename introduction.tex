\section{Introduction}
\label{sec:intro}

In recent years the term Big Data has become a phenomenon that breaks down borders of many technologies and approaches that have so far been acknowledged as mature and robust for any conceivable application. Gartner Inc.\footnote{\url{http://www.gartner.com/}} describes Big Data as ``\emph{high \textbf{v}olume, high \textbf{v}elocity, and/or high \textbf{v}ariety information assets that require new forms of processing to enable enhanced decision making, insight discovery and process optimization}''.

...

\TODO{there are two types of graph big data sets - large graphs and large sets of small graphs; we focus on the former one}

\paragraph*{Contributions} In this survey we will focus on graph data generators in the context of Big Data and related applications of graph algorithms. The key contributions are as follows:
\begin{itemize}
  \item ...
\end{itemize}

...

\paragraph*{Differences with prior surveys}

To the best of our knowledge, our work is the first surveying the broad landscape of graph data generators spanning different Big Data applications and 
targeting diverse computer science subfields. In particular, we cover graph database
generators, graph processing generators, social networks and community detection
generators, Semantic Web data generators, graph
analytics generators, IOT, Telecommunication and graph streaming generators, 
and Machine Learning and graph mining generators. A comprehensive study of
graph generators is missing for many of the specific subfields mentioned
above.   

A limited subset of graph database
generators, parallel and distributed graph processing generators, along
with a few of the Semantic Web data generators presented in our survey have
been discussed in a related book chapter
\cite{BFHI18} while cross-comparing them with respect to input, output,
supported workload, data model and query language, along with the
distinguished chokepoints. However, the provided classification is
inherently data-driven and not meant to serve the purpose of letting any
researcher or practitioner interested in Big Data be able to make a guided
choice of the desired 
generator based on its functional and goal-driven features (such as the application domain,
the supported operations and the key configuration options). 
Moreover, our work is much broader and targets graph generators of several 
diversified communities, not limiting its scope to few generators of the database
and graph processing communities. 

Graph generators matching graph patterns used in data mining have been
studied in \cite{Chakrabarti:2006:GML:1132952.1132954}, 
focusing on mostly occurring patterns, such as power laws, size of graph diameters
and community structure. The considered graph generators are compared in
terms of graph type, degree distributions, exponentiality, diameter and
community effects. We refer the reader to this survey for taxonomies
involving these properties, whereas we provide here a functionality-driven
taxonomy across all the categories of graph generators that we consider.    
We also point out that this survey is outdated as it does not consider the 
graph mining generators that fervently appeared in the last decade. 

\TODO{Check with Arnau}

Aggarwal and Subbian \cite{AggarwalS14} have surveyed evolution analysis in
graphs, by primarily focusing on data mining maintenance methods and on analytical 
quantification and explanation of the changes of the underlying networks. 
A brief discussion on evolutionary
network data generators is carried out in
the paper. The data generation of evolutionary networks based on 
Densification Power Law (DPL) and shrinking diameters \cite{LeskovecKF05} and community-guided
attachment properties \cite{LeskovecKF05} is then considered, along with tackling Kronecker
recursion with recursive tensor multiplication \cite{AkogluMF08}.
 
\TODO{Better highlight the differences when Sherif's part is in.}

%\TODO{Outreach other surveys in other communities.}


\paragraph*{Outline} The rest of the text is structured as follows: ...
% In Section~\ref{sec:preliminaries} we provide a brief introduction to graph theory and related terms.
% In Section~\ref{sec:classification} we classify the current Big graph Data applications.
% In Section~\ref{sec:generators} we introduce the existing graph data generators in the context of the proposed categories and their main features.
% Section~\ref{sec:comparison} provides comparison of the described generators.
% In Section~\ref{sec:challenges} we discuss challenges and open problems. And in Section~\ref{sec:conclusion} we conclude.


