\section{Introduction}
\label{sec:intro}

%In recent years the term Big Data has become a phenomenon that breaks down borders of many technologies and approaches that have so far been acknowledged as mature and robust for any conceivable application. Gartner Inc.\footnote{\url{http://www.gartner.com/}} describes Big Data as ``\emph{high \textbf{v}olume, high \textbf{v}elocity, and/or high \textbf{v}ariety information assets that require new forms of processing to enable enhanced decision making, insight discovery and process optimization}''.

%\TODO{The abundance of interconnected data has fueled the design and implementation of graph generators reproducing real-world linking properties, or gauging the effectiveness of graph algorithms, techniques and applications manipulating these data.}

%One of the most popular data structures in the world of computer science are graphs.
Graphs are ubiquitous data structures that span a wide array of scientific disciplines
and are a subject of study in several subfields of computer science.
Nowadays, due to the dawn of numerous tools and applications manipulating
graphs, they are adopted as a rich data model in many data-intensive use
cases involving disparate domain knowledge, mathematical foundations and
computer science. Interconnected data is oftentimes used to encode domain-specific
use cases, such as recommendation networks, social networks,
protein-to-protein interactions, geolocation networks, and fraud detection analysis networks, to
name a few.
% ANGELA: I do not think we should cite Neo4J, which is a graph database.
%They enable to represent complex information in a simple and yet powerful way supported by a strong mathematical background. With the dawn of the new millennium which brought numerous novel technologies and applications, the graphs have become even more popular. There has appeared a number of use cases\footnote{\url{https://neo4j.com/use-cases/}} (such as fraud detection, recommendation engines, social networks etc.) where graphs represent the linked information.

In general, we can distinguish between two broad types of graph data sets: (1) a single large graph (possibly with several components), such as social networks  or Linked Data graphs, and (2) a large set of small graphs, such as chemical compounds\footnote{\url{https://pubchem.ncbi.nlm.nih.gov/}} or linguistics syntax trees\footnote{\url{https://catalog.ldc.upenn.edu/ldc99t42}}. Naturally, the algorithms used in theses two classes differ a lot~\cite{DBLP:books/igi/Sakr2011}. In the former case we can search, e.g., for communities and their features or shortest paths, while in the latter case we usually query for supergraphs, subgraphs, or graphs similar to a given graph pattern. In both the cases, as in the other fields, quite often the respective real-world graph data is not publicly available (or simply does not exist when a particular method for data manipulation is proposed).
Even in the cases in which real data is abundant, many algorithms and techniques need to be tested on various orders of magnitudes of the graph sizes thus leading to the inception of
configurable graph generators that reproduce real-world graph properties and provide unique tuning opportunities for
algorithms and tools handling such data.
%As a consequence, also in the world of graphs there has appeared a number of graph data generators which enable to  reproduce the respective real-world linking properties and gauge the effectiveness of a particular algorithm.

In this survey, we provide a comprehensive overview of the state-of-the-art
modern
graph generators by focusing on those that are pertinent and suitable for
data-intensive tasks. Our aim is to cover a wide range of currently popular
areas of graph data processing. In particular, we consider graph generation
in the areas of Semantic Web, graph databases, social networks, and community detection, along with general graphs. Despite the disparate requirements of modern graph generators throughout these communities, we analyze them under a common umbrella, reaching out the functionalities, the practical usage, and the supported operations of graph generators. The reasons for this scope and classification are as follows:

\begin{enumerate}
  \item Despite the differences of the covered areas, the requirements for
modern graph data generators can be similar in particular cases. Reusing or learning from tools in other fields can thus bring new opportunities for both researchers and practitioners.
  \item The selected classification is serving the need of providing scientists, researchers, and practitioners with the right data generator at hand for their work.
\end{enumerate}

To conclude the comparative study and provide a comprehensive view of the
field,  we also overview the most popular real-world data sets used in the respective covered areas and discuss open challenges and missing requirements of modern
graph generators in view of identifying their future extensions to new emerging fields.


\paragraph*{Contributions} Our survey revisits the representatives of
modern graph data generators and summarizes their major features. The comprehensive review and analysis make this paper useful for motivating new graph data generation techniques as well as serving as a technical reference for selecting suitable solutions. In particular, the introductory categorization and comparative study enables the reader to quickly get his/her bearings in the field and identify a subset of generators of his/her interest. Since we do not limit the survey to a particular area of graph data management,  the reader can get a broader scope in the field. Hence, while practitioners can find a solution in another previously unexpected area, researchers can identify new target areas or exploit successful results in new fields. Last but not least, %to provide a global view of the state-of-the-art, we also provide an overview of most commonly used real-world testing graph datasets. And, finally,
we identify general open problems of graph data synthesis and indicate possible solutions to show that it still forms a challenging and promising research area.

\paragraph*{Differences with prior surveys}

To the best of our knowledge, our work is the first to survey the broad
landscape of graph data generators spanning different data-intensive applications and
targeting many computer science subfields. In particular, we cover Semantic Web, graph databases, social networks, and community detection, along with general graphs. However, the literature is still lacking a comprehensive study of graph generators  for many of the specific subfields mentioned above.

A limited subset of graph database
generators, parallel and distributed graph processing generators, along
with a few of the Semantic Web data generators presented in our survey have
been discussed in a related book chapter
\cite{BFHI18} while cross-comparing them with respect to input, output,
supported workload, data model and query language, along with the
distinguished chokepoints. However, the provided classification is
inherently database-oriented.
In our work, we provide a more comprehensive and broader classification
 that serves the purpose of letting any
researcher or practitioner interested in data generation  to be able to make a better
choice of the desired graph
generator based on its functional and goal-driven features (such as the application domain,
the supported operations, and the key configuration options).
Moreover, in contrast with \cite{BFHI18}, our work encompasses graph generators of several
diverse communities, not limiting its scope to a few generators of the database
and graph processing communities.

Graph generators matching graph patterns used in data mining have been
studied in \cite{Chakrabarti:2006:GML:1132952.1132954},
focusing on mostly occurring patterns, such as power laws, size of graph diameters,
and community structure. The considered graph generators are compared in
terms of graph type, degree distributions, exponentiality, diameter, and
community effects. We refer the reader to this survey for taxonomies
involving these properties, whereas we provide here a functionality-driven
taxonomy across all the categories of graph generators that we consider.
We also point out that this survey is outdated as it does not consider the
generators that appeared in the last decade.
We fill the gap of more recent social network generators
in Section~\ref{sec:generators_socialnetworks}, as well as more recent representatives of the other categories.

%\TODO{Check with Arnau}

Aggarwal and Subbian~\cite{AggarwalS14} have surveyed the evolution of analysis in
graphs, by primarily focusing on data mining maintenance methods and on analytical
quantification and explanation of the changes of the underlying networks.
A brief discussion on evolutionary
network data generators is carried out in
the paper. The data generation of evolutionary networks is based on the
Densification Power Law (DPL) and shrinking diameters, community-guided
attachment properties~\cite{Leskovec:2005:GOT:1081870.1081893}.
Generation of graphs tackling Kronecker
recursion with recursive tensor multiplication~\cite{AkogluMF08} is then considered in the above survey.
We refer the readers to the aforementioned survey for evolutionary network generators and further 
discuss the open challenges of evolving graph data in Section~\ref{sec:challenges}.
%\TODO{Better highlight the differences when Sherif's part is in.}

%\TODO{Outreach other surveys in other communities.}


\paragraph*{Outline} The rest of this paper is organized as follows: Section~\ref{sec:comparison}  
provides the opening categorization and comparison of the generators. Section~\ref{sec:generators} 
provides an overview of the existing graph data generators and their main features in the frame of 
the proposed categories. 
%To provide a complete overview,  we list the most commonly used testing data sets for the particular categories in Section~\ref{sec:data_sets}.
In Section~\ref{sec:challenges}, we highlight some of the challenges and open problems of graph 
data synthesis before we conclude in Section~\ref{sec:conclusion}.


