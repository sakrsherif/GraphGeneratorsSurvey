\subsection{Testing Community Detection}
\label{sec:generators_community_detection}

Community detection is one of the many graph analytics algorithms typically used
in domains such as social networks or bioinformatics. \emph{Communities} are groups of nodes that are highly connected among them, while being scarcely connected to
the rest of the graph. Such communities emerge from the fact that real graphs
are not random, but follow real-world dynamics that make similar entities to
have a larger probability to be connected. As a consequence, detected
communities are used to reveal vertices with similar characteristics, for
instance to discover functionally equivalent proteins in protein-protein
interaction networks, or persons with similar interests in social networks. Such
applications have made community detection a hot topic during the last fifteen
years with tens of developed algorithms and detection
strategies~\cite{doi:10.1002/wics.1403,Kim:2015:CDM:2854006.2854013}. For
comparing the quality of the different proposed techniques, one needs graphs
with \emph{reference communities}, that is, communities known beforehand. Since
it is very difficult to have large real graphs with reference communities
(mainly because these would require a manual labeling), graphs for benchmarking
community detection algorithms are typically generated synthetically.

\paragraph{Danon et al.} The first attempts to compare community detection algorithms using synthetic
graphs proposed the use of random graphs composed by several Erd\"{o}s-R\'{e}nyi
subgraphs, connected more internally than externally~\cite{danon2005comparing}.
Each of these subgraphs has the same size and the same internal/external density
of edges. However, such graphs miss the realism observed in real graphs, where
communities are of different sizes and densities, thus several proposals exist
to overcome such an issue.

\paragraph{LFR} Lancichinetti, Fortunato
and Radicchi (hence LFR)~\cite{PhysRevE.78.046110} propose a class of benchmark
graphs for community detection where communities are of diverse sizes and
densities. The generated communities follow a power-law distribution whose parameters can be configured. The degree of the
nodes is also sampled from a power-law distribution. Additionally, the generator
introduces the concept of the ``mixing factor'', which consists of the percentage of
edges in the graph connecting nodes that belong to different communities. Such parameter
allows  the degree of modularity of the generated graph  to be tuned, thus
testing the robustness of the algorithms under different conditions. The
generation process is implemented as an optimization process starting with an empty graph and
progressively filling it with nodes and edges guided by the specified constraints.

\paragraph{LFR-Overlapping} Lancichinetti, Fortunato and
Radicchi~\cite{PhysRevE.80.016118} extended LFR to support the notion of
directed graphs and overlapping communities. Overlapping communities extend the
notion of communities by allowing the sharing of vertices, thus a vertex can
belong to more than one community. This extended generator allows controlling
the same parameters of LFR, as well as the amount of overlap of
the generated communities.

\paragraph{Strengths and Weaknesses of Community Detection Generators}
Besides synthetic graph generators, Yang and Leskovec~\cite{yang2015defining}
proposed the use of real-world graphs with explicit group annotations (e.g.,
forums in a social network, categories of products, etc.) to infer what they
call ``meta-communities'', and use them to evaluate overlapping community
detection algorithms. However, a recent study from Hric, Darst and
Fortunato~\cite{hric2014community} reveal a loose correspondence between
communities (the authors refer to them as structural communities) and
meta-communities.  This result reveals that  algorithms working for structural
communities do not work well for finding meta-communities and vice versa,
suggesting significantly different underlying characteristics
between the two types of communities, which are yet to be
identified.

In this regard and to the best of our knowledge, there does not exist a set of
generators to generate graphs with meta-communities for community detection
algorithm benchmarking. The closest one is the LDBC-SNB data
generator~\cite{Erling:2015:LSN:2723372.2742786} which has been provided by the
generation of groups of users in the social network. Even though the generation
process does not specifically enforce the generation of groups
(meta-communities) for benchmarking community detection algorithms, the study
from Prat-Pe\'rez and Dominguez-Sal reveals that these groups are more similar
to the real meta-communities than those structural communities generated by the
LFR benchmark~\cite{Prat-Perez:2014:CSS:2621934.2621942}.

The differences observed between structural and meta-communities reveal the need
of more accurate community definitions tight more specifically to the domain or
use case. Current community detection algorithms and graph generators for
community detection are stuck to the traditional (and vague) definition of
community, assuming that there exists a single algorithm that would fit all the
use cases. Thus, future work requires the study of domain-specific community
characteristics that can be used to generate graphs with a community structure
that accurately resembles that of specific use cases, and thus revealing which
are the best algorithms for each particular scenario.

