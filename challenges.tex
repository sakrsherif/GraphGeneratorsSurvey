\section{Challenges and Open Problems}
\label{sec:challenges}

...

\subsection{Single- vs. multi-domain}
...

\subsection{Simple Usage, Simple Parameters}
Proposal of a data generator (not necessarily for graph data) has to face an important schism. On the one hand, there is the aim to provide the user with as many parameters as possible in order to enable one to generate any kind of data. This approach seems to be reasonable, but it encounters the fact that users are usually unwilling to use complex benchmarking tools. This observation can be seen, for example, in the case of XML benchmarks -- even though there exist robust and complex data generators (such as ToXGene~\cite{conf/webdb/BarbosaMKL02}, which supports specification of structural aspects, value distributions, references etc.), the most popular benchmarking tool is XMark~\cite{Schmidt:2002:XBX:1287369.1287455} which models a single use case and enables to specify just the size of the data. Hence, the other extreme is to provide a simple data generator which does not require any complex settings and thus the benchmarking process is extremely simple and fast.

Considering the complex structure of graph data and the variety of applications having highly specific types of graphs, the latter solution is difficult to implement. A reasonable compromise can thus be seen in a data generator which is provided with sample graph data and is capable of automatic analysis of their structural and value specifics in order to gain the complex parameters.

\subsection{Evolving Graph Data}
As user requirements as well as environments change, most of the existing applications naturally evolve over time, at least to some extent. This evolution usually influences the structure of the data and consequently all the related parts of the application (i.e., storage strategies, operations, indices etc.). The respective data generator should hence be able to simulate such a growth and/or change in structure.

In some graph applications, such as, e.g., social networks, the evolution of the data is a significant aspect, especially in the activity graphs, that has been studied extensively~\cite{doreian1997evolution,Kumar:2006:SEO:1150402.1150476,Hellmann2014583,wang2013,Kossinets88,Viswanath:2009:EUI:1592665.1592675}.  A related problem is \emph{data versioning} and respective ability to query across multiple versions of data or in general its analysis. Considering graph data this problem is also common, for example  in the area of Linked Data~\cite{DBLP:conf/semweb/Papakonstantinou16,DBLP:conf/esws/MeimarisP16,fernandez2015towards,fernandez2015bear}.

As shown in papers~\cite{Leskovec:2005:RMT:2101235.2101254,Leskovec:2005:GOT:1081870.1081893}, evolving graphs have further specific features. For example, some graphs grow over time according to a \emph{densification power law} which means that real graphs tend to sprout many more edges than nodes, and thus are densifying as they grow. Also the effective diameter of graphs tends to shrink or stabilize as the graph grows with time.

%\paragraph{} Following the same recursion idea, paper~\cite{Leskovec:2005:RMT:2101235.2101254} uses Kronecker multiplication to generate self-similar graphs. The network starts with an initial graph G1 that contains $N_1$ nodes and $E_1$ edges. Using matrix recursion, larger successive graphs $G_2, G_3, ... G_n$ are generated. The $k$-th graph $G_k$ contains $N_k = N^k_1$ nodes. Many graphs often densify over time, exhibiting a growth in the number of edges that is superlinear to the number of nodes. Kronecker multiplication produces graphs with a fixed diameter and a densification power law degree distribution with exponent $k = log(E_1)/log(N_1)$. The graph generation process introduces a staircase effect in the nodes' degrees, and each community consists of smaller nested communities that are formed through expansion and recursion.


\subsection{Multi-model Data}
With the dawn of Big Data and especially its Variety aspect there have emerged also new types of database management systems. One of the most interesting ones are so-called \emph{multi-model databases} which enable to store and especially query across structurally different data. There exist various types of multi-model systems combining  distinct subsets of Big Data structures; hence, there naturally exists systems which combine graph data with other data models. For example, OrientDB\footnote{\url{http://orientdb.com/orientdb/}} which was implemented on the basis of an object DBMS currently supports graph, document, key/value, and object models.

Such type of DBMSs also needs a specific benchmark/data generator which enables to test new features and analyze efficiency of operations. However, since the multi-model systems are in the context of Big Data rather new, there exist only a few benchmarks targeting multi-model DBMSs (such as Bigframe~\cite{journals/pvldb/KunjirKB14} or UniBench~\cite{conf/cidr/lu17}) with limited capabilities.

\subsection{Community Detection}
\label{sec:generators_community_detection}

Community detection is one of the many graph analytics algorithms typically used
on domains such as social networks or bioinformatics. Communities are groups of nodes that are highly connected among them, while being scarcely connected to
the rest of the graph. Such communities emerge from the fact that real graphs
are not random, but follow real-world dynamics that make similar entities to
have a larger probability to be connected. As a consequence, detected
communities are used to reveal vertices with similar characteristics, for
instance to discover functionally equivalent proteins in protein-protein
interaction networks, or persons with similar interests in social networks. Such
applications have made community detection a hot topic during the last fifteen
years with tens of developed algorithms and detection
strategies~\cite{doi:10.1002/wics.1403,Kim:2015:CDM:2854006.2854013}. For
comparing the quality of the different proposed techniques, one needs graphs
with \emph{reference communities}, that is, communities known beforehand. Since
it is very difficult to have large real graphs with reference communities
(mainly because these would require a manual labeling), graphs for benchmarking
community detection algorithms are typically generated synthetically.

\paragraph{Danon et al.} The first attempts to compare community detection algorithms using synthetic
graphs proposed the use of random graphs composed by several Erd\"{o}s-R\'{e}nyi
subgraphs, connected more internally than externally~\cite{danon2005comparing}.
Each of these subgraphs has the same size and the same internal/external density
of edges. However, such graphs miss the realism observed in real graphs, where
communities are of different sizes and densities, thus several proposals exist
to overcome such an issue.

\paragraph{LFR} Lancichinetti, Fortunato
and Radicchi (hence LFR)~\cite{PhysRevE.78.046110} propose a class of benchmark
graphs for community detection where communities are of diverse sizes and
densities. The generated communities follow a power-law distribution whose parameters can be configured. The degree of the
nodes is also sampled from a power-law distribution. Additionally, the generator
introduces the concept of the ``mixing factor'', which consists of the percentage of
edges in the graph connecting nodes that belong to different communities. Such parameter
allows  the degree of modularity of the generated graph  to be tuned, thus
testing the robustness of the algorithms under different conditions. The
generation process is implemented as an optimization process starting with an empty graph and 
progressively filling it with nodes and edges guided by the specified constraints.

\paragraph{LFR-Overlapping} Lancichinetti, Fortunato and
Radicchi~\cite{PhysRevE.80.016118} extended LFR to support the notion of
directed graphs and overlapping communities. Overlapping communities extend the
notion of communities by allowing the sharing of vertices, thus a vertex can
belong to more than one community. This extended generator allows controlling
the same parameters of LFR, as well as the amount of overlap of
the generated communities.

\paragraph{Strengths and Weaknesses of Community Detection Generators}
Besides synthetic graph generators, Yang and Leskovec~\cite{yang2015defining}
proposed the use of real-world graphs with explicit group annotations (e.g.,
forums in a social network, categories of products, etc.) to infer what they
call ``meta-communities'', and use them to evaluate overlapping community
detection algorithms. However, a recent study from Hric, Darst and
Fortunato~\cite{hric2014community} reveal a loose correspondence between
communities (the authors refer to them as structural communities) and
meta-communities.  This result reveals that  algorithms working for structural
communities do not work well for finding meta-communities and vice versa,
suggesting significantly different underlying characteristics
between the two types of communities, which are yet to be
identified.

In this regard and to the best of our knowledge, there does not exist a set of
generators to generate graphs with meta-communities for community detection
algorithm benchmarking. The closest one is the LDBC-SNB data
generator~\cite{Erling:2015:LSN:2723372.2742786} which has been provided by the
generation of groups of users in the social network. Even though the generation
process does not specifically enforce the generation of groups
(meta-communities) for benchmarking community detection algorithms, the study
from Prat-Pe\'rez and Dominguez-Sal reveals that these groups are more similar
to the real meta-communities than those structural communities generated by the
LFR benchmark~\cite{Prat-Perez:2014:CSS:2621934.2621942}.

The differences observed between structural and meta-communities reveal the need
of more accurate community definitions tight more specifically to the domain or
use case. Current community detection algorithms and graph generators for
community detection are stuck to the traditional (and vague) definition of
community, assuming that there exists a single algorithm that would fit all the
use cases. Thus, future work requires the study of domain-specific community
characteristics that can be used to generate graphs with a community structure
that accurately resembles that of specific use cases, and thus revealing which
are the best algorithms for each particular scenario.



