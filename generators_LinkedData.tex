\subsection{Semantic Web}
\label{sec:generators_LinkedData}
With the dawn of the concept of Linked Data it is a natural development that there would emerge respective
benchmarks involving both synthetic data and  data sets   with real-world characteristics.
The used data sets correspond to RDF representation of relational-like data~\cite{Guo2005158,Bizer09theberlin}, social network-like data~\cite{Schmidt2010}, or specific and significantly more complex data structures such as biological data~\cite{Wu2014}. In this section, we provide an overview of benchmarking systems involving a kind of graph-based RDF data generator or data modifier. %Other types of systems or particular results can be found, e.g., at~\cite{RdfStoreBenchmarking}.


\iffalse
Considering the Big Data world, the Linked Data in general definitely belong to this group since we assume that the Linked (Open) Data Sets form a common Linked Open Data cloud\footnote{\url{http://lod-cloud.net/}}. On the other hand, the particular data sets can be relatively small.
\fi

\paragraph{LUBM} The use-case driven Lehigh University Benchmark (LUBM)\footnote{\url{http://swat.cse.lehigh.edu/projects/lubm/}} considers the university domain. The ontology defines 43 classes and 32 properties~\cite{Guo2005158}. In addition, 14 test queries are provided in the LUBM benchmark. In particular, the benchmark focuses on \emph{extensional} queries, i.e., queries which target the particular data instances of the ontology, as an opposite to \emph{intentional} queries, i.e., queries which target properties and classes of the ontology. The Univ-Bench Artificial  (UBA) data generator features repeatable and random  data generation (exploiting classical linear congruential generator, LCG, of numbers). In particular, the data which is produced by the generator are assigned zero-based indexes (i.e., \emph{University0}, \emph{University1} etc.), thus they are reproducible at any time with the same indexes.  The generator naturally allows to specify a seed for random number generation, along with the starting index and the desired number of universities.

An extension of LUBM, the Lehigh BibTeX Benchmark (LBBM)~\cite{Wang2005}, enables generating synthetic data for different ontologies. The data generation process is managed through two main phases: (1) the property-discovery phase, and (2) the data generation phase. LBBM provides a probabilistic model that can emulate the discovered properties of the data of a particular domain and generate synthetic data exhibiting similar properties. Synthetic data are generated using a Monte Carlo algorithm. The approach is demonstrated on the Lehigh University BibTeX ontology which consists of 28 classes along with 80 properties. The LUBM benchmark includes 12 test queries that were designed for the benchmark data. Another extension of LUBM, the University Ontology Benchmark (UOBM)\footnote{\url{https://www.cs.ox.ac.uk/isg/tools/UOBMGenerator/}}, focuses on two aspects: (1) usage of all constructs of OWL Lite and OWL DL~\cite{owl} and (2) lack of necessary links between the generated data which thus form isolated graphs~\cite{Ma:2006:TCO:2094613.2094629}. In the former case the original ontology is replaced by the two types of extended versions from which the user can choose. In the latter case cross-university and cross-department links are added to create a more complex graph.

\paragraph{IIMB} Ferrara et al.~\cite{Ferrara08OM} proposed the ISLab Instance Matching Benchmark (IIMB)\footnote{\url{http://www.ics.forth.gr/isl/BenchmarksTutorial/}} for the problem of instance matching. For any two objects $o_1$ and $o_2$ adhering to different ontologies or to the same ontology, instance matching is specified in the form of a function $Om(o_1, o_2) \rightarrow \{0, 1\}$,  where $o_1$ and $o_2$ are linked to the same real-world object (in which case the function maps to $1$) or $o_1$ and $o_2$ are representing different objects (in which case the function maps to $0$). It targets the domain of movie data which contains 15 named classes, along with 5 objects and 13 datatypes. The data are extracted from IMDb\footnote{\url{http://www.imdb.com/}}. The data generator corresponds to a data modifier which simulates differences between the data. In particular it involves data value differences (such as typographical errors or usage of different standard formats, e.g., for names), structural heterogeneity (represented by different levels of depth for properties, diverse aggregation criteria for properties, or missing values specification) and logical heterogeneity (such as, e.g., instantiation on disjoint classes or various subclasses of the same superclass).


\paragraph{BSBM} The Berlin SPARQL Benchmark (BSBM)\footnote{\url{http://wifo5-03.informatik.uni-mannheim.de/bizer/berlinsparqlbenchmark/}}, is centered around an e-commerce application domain with object types such as \emph{Customer}, \emph{Vendor}, \emph{Product} and \emph{Offer} in addition to the relationship among them~\cite{Bizer09theberlin}.
The benchmark provides a workload that has 12 queries with 2 types of query workloads (i.e., 2 sequences of the 12 queries) emulating the navigation pattern and search of a consumer seeking a product. The data generator is capable of producing arbitrarily scalable datasets by controlling the number of products ($n$) as a scale factor.  The scale factor also impacts other data characteristics, such as, e.g., the depth of type hierarchy of products, branching factor, the number of product features,  etc. BSBM can output two representations, i.e. an RDF representation along with a relational representation. Thus, BSBM also defines an SQL~\cite{sql} representation of the queries. This allows comparison of SPARQL~\cite{sparql} results  to be compared against the performance of traditional RDBMSs.


\paragraph{SP$^2$Bench} The SP$^2$Bench\footnote{\url{http://dbis.informatik.uni-freiburg.de/forschung/projekte/SP2B/}} 
is a language-specific benchmark~\cite{Schmidt2010} which is based on the DBLP dataset. %, so the types involve Person, Inproceedings, Article etc.
The generated datasets follow the key characteristics of the original DBLP dataset. In particular, the data mimics the correlations between entities. All random functions of the generator use a fixed seed that ensures that the data generation process is deterministic. SP$^2$Bench is accompanied by 12 queries covering the various types of operators such as RDF access paths in addition to typical RDF constructs.

%\paragraph{JustBench} ~\cite{Bail:2010:JFO:1940281.1940285} ...




\paragraph{DBPSB} DBpedia SPARQL Benchmark (DBPSB)\footnote{\url{http://aksw.org/Projects/DBPSB.html}} proposed at the University of Leipzig has been designed using workloads that have been generated by applications and humans~\cite{Morsey2011,Morsey:2012:UBR:2900929.2901031}. In addition, the authors argue that benchmarks like LUBM, BSBM, or SP$^2$Bench resemble relational database benchmarks involving relational-like data which is structured using a small amount of homogeneous classes, whereas, in reality, RDF datasets are tending to be more heterogeneous. For example, DBpedia 3.6 consists of 289,016 classes, whereas 275 of them are defined based on the DBpedia ontology. In addition, in property values different data types as well as references to objects of the various types are
used. Hence, they presented a universal SPARQL benchmark generation approach which uses a flexible data production mechanism that mimics the input data source. This dataset generation process begins using an input dataset; then multiple datasets with different sizes  are then generated by duplicating all the RDF triples with changing their namespaces.  For generating smaller datasets, an adequate selection of all triples is selected randomly or using a sampling mechanism over the various classes in the dataset. \iffalse The goal of the query analysis and clustering is to detect prototypical queries on the basis of their frequent usage and similarity.\fi The methodology is applied on the DBpedia SPARQL endpoint and a set of 25 templates of SPARQL queries is derived to cover frequent SPARQL features.

\paragraph{LODIB} The Linked Open Data Integration Benchmark (LODIB)\footnote{\url{http://lodib.wbsg.de/}} has been designed with the aim of reflecting the real-world heterogeneities that exist on the Web of Data in order to enable testing of Linked Data translation systems~\cite{DBLP:conf/www/RiveroSBR12}. It provides a catalogue of 15 data translation patterns (e.g., rename class, remove language tag etc.), each of which is a common data translation problem in the context of Linked Data. The benchmark provides a data generator that produces three different synthetic data sets that need to be translated
by the system under test into a single target vocabulary. They  reflect the pattern distribution in analyzed 84 data translation examples from the LOD Cloud. The data sets reflect the same e-commerce scenario used for BSBM.




\paragraph{Geographica} The Geographica benchmark\footnote{\url{http://geographica.di.uoa.gr/}} has been designed to target the area of geospatial data~\cite{DBLP:conf/semweb/GarbisKK13} and respective SPARQL extensions GeoSPARQL~\cite{battle2012enabling} and stSPARQL~\cite{koubarakis2010modeling}. The benchmark involves a real-world workload that uses openly available datasets that cover various geometry elements (such as, e.g., lines, points, polygons, etc.) and  a synthetic workload. In the former case there is a (1) a micro benchmark that evaluates primitive spatial functions (involving 29 queries) and (2) macro benchmark that tests the performance of RDF engines in various application scenarios such as  map exploring and search (consisting of 11 queries). In the latter case of a synthetic workload the generator produces synthetic datasets of different sizes that corresponds to an ontology based on OpenStreetMap  and instantiates query templates. \iffalse The spatial extent of the land ownership dataset constitutes a uniform grid of $n \times n$ hexagons, whereas the size of each dataset is given relatively to $n$.\fi The generated SPARQL query workload is corresponding to spatial joins and selection using 2 query templates.


\paragraph{WatDiv} The Waterloo SPARQL Diversity Test Suite (WatDiv)\footnote{\url{http://dsg.uwaterloo.ca/watdiv/}} has 
been designed at the University of Waterloo. It implements stress testing tools that focus on addressing the observation 
that the state-of-the-art SPARQL benchmarks do not fully cover the variety of queries and  
workloads~\cite{Aluc:2014:DST:2717213.2717229}. The benchmark focuses on two types of query aspects -- 
structural and data-driven -- and performs a detailed analysis on existing SPARQL benchmarks 
(LUBM, BSBM, DBPSB, and SP$^2$Bench) using these two properties of queries. The structural features involve triple 
pattern count, join vertex count, join vertex degree, and join vertex count. The data-driven features involve 
result cardinality and several types of selectivity. The analysis of the four benchmarks reveals that their diversity is
insufficient for evaluation of the weaknesses/strengths of the distinct design alternatives implemented by the different 
RDF systems. 

In particular, WatDiv, provides (1) a data generator which generates scalable datasets according to the WatDiv 
schema, (2) a query template generator which produces a  set of query templates according to the WatDiv schema, and 
(3) a query generator that uses the generated templates and instantiates them  with real RDF values from the dataset, and (4) a feature extractor which extracts the structural features of the generated data and workload. %For the study in the paper the authors generated 12,500 test queries from 125 query templates.

\paragraph{RBench} RBench~\cite{Qiao:2015:RAR:2723372.2746479} is an application-specific benchmark which receives any RDF dataset as an input and produces a set of datasets, that have similar characteristics of the input dataset, using size scaling factor $s$ and (node) degree scaling factor $d$. These factors ensure that the original RDF graph $G$ and the synthetic graph $G'$ are similar and the average node degree and the number of edges of $G'$ are changed by $s$ and $d$ respectively. \iffalse A generated benchmark dataset is considered similar to the given dataset if their values for the dataset evaluation metrics and query evaluation times for different techniques are similar. Three evaluation metrics are utilized for this purpose: dataset coherence (i.e., a measure how uniformly predicates are distributed among the same type/class), relationship specialty (i.e., the number of occurrences of the same predicate associated with each resource), and literal diversity.\fi A query generation process has been implemented to produce 5 different types of queries (edge-based queries, node-based queries, path queries, star queries, subgraph queries) for any generated data. The benchmark project FEASIBLE~\cite{Saleem2015} is also an application-specific benchmark; however, contrary to RBench, it is designed to produce benchmarks from the set of sample input queries of a user-defined
size.

In practice, one way for handling big RDF graphs is to process them using the
\emph{streaming} mode where the data stream could consist of the edges of the
graph. In this mode, the RDF processing algorithms can process the input
stream in the order it arrives while using only a limited amount of
memory~\cite{mcgregor2014graph}. The streaming mode has mainly  attracted the attention of the
RDF and Semantic Web community.

\paragraph{S2Gen}   Phuoc et al.~\cite{le2012linked} presented
an evaluation framework for linked stream data processing engines. The framework
uses a datasets generated with the Stream Social network data Generator
(S2Gen), which
simulates streams of user interactions (or events) in social networks
(e.g., posts) in addition to the  user metadata such as users' profile
information, social network relationships, posts, photos and GPS information.
The data generator of this framework provides the users the flexibility to
control the characteristics of the generated stream by tanning a range of
parameters, which includes the frequency at which interactions are generated,
limits such as the maximum number of messages per user
and week, and the correlation probabilities between the different objects (e.g.,
users) in the social network.

\paragraph{RSPLab} Tommasini et al.~\cite{tommasini2017rsplab} introduced
another framework for benchmarking RDF Stream Processing systems, RSPLab. The
Streamer component of this framework is designed to publish RDF streams from the
various existing RDF benchmarks (e.g., BSBM, LUBM).
In particular, the Streamer  component uses TripleWave\footnote{\url{http://streamreasoning.github.io/TripleWave/}}, an
open-source framework which enables to share RDF streams on the
Web~\cite{mauri2016triplewave}.   TripleWave acts as a means for plugging-in
and combining streams from multiple Web data sources using either pull or push mode.



\paragraph{LDBC}  The Linked Data Benchmark Council\footnote{\url{http://ldbcouncil.org/industry/organization/origins}} (LDBC)~\cite{Angles:2014:LDB:2627692.2627697} %is a result of a (closed) EU project that brought together a community of academic researchers and industry that
had the goal of developing an open source, yet industrial grade benchmarks for RDF and graph databases. 
\iffalse The following three benchmarks were developed and are currently maintained.\fi In the Semantic Web 
domain, it released the Semantic Publishing Benchmark (SPB)~\cite{spb} that has been inspired by the 
Media/Publishing industry (namely BBC\footnote{\url{http://www.bbc.com/}}). The application scenario 
of this benchmark simulates a media or a publishing organization that handles large amount 
of streaming content (e.g., news, articles). \iffalse This content is enriched with metadata that describes 
it and links it to reference knowledge -- taxonomies and databases that include relevant concepts, 
entities and factual information. The SPB data generator produces scalable in size synthetic large data. 
Synthetic data consists of a large number of annotations of media assets that refer entities found in 
reference datasets.\fi The data generator mimics three types of relations in the generated synthetic data: 
correlations of entities, data clustering, and random tagging of entities. Two workloads are provided: (1) basic, involving an interactive query-mix querying the relations between entities in reference data, and (2) advanced,  focusing on interactive and analytical query-mixes. The LDBC has designed two other benchmarks: the Social Network Benchmark (SNB)~\cite{Erling:2015:LSN:2723372.2742786} for the social network domain  (see Section~\ref{sec:generators_socialnetworks}) and Graphalytics~\cite{Iosup:2016:LGB:3007263.3007270}   for the analytics domain.% (see Section~\ref{sec:generators_analytics}).



\paragraph{LinkGen} LinkGen is a synthetic linked data generator that has been designed to generate RDF datasets for a given vocabulary~\cite{10.1007/978-3-319-46547-0_12}. The generator is designed to receive a vocabulary as an input  and supports two statistical distributions for generating entities:  Zipf's power-law distribution and Gaussian distribution. LinkGen can augment the generated data with inconsistent and noisy  data such as updating a given datatype property with two conflicting values or  adding triples with syntactic errors. \iffalse, adding wrong statements by assigning them with invalid domain and creating instances with no type information.\fi The generator also provides a feature to inter-link the generated objects with real-world ones from user-provided real-world datasets. The datasets can be generated in any of of two modes: on-disk and streaming.


\paragraph{Strengths and Weaknesses of Semantic Web Graph Generators}  Graphs are intuitive and standard representation for the RDF model that form the basis for the Semantic Web community which has been very active on building several benchmarks, associated with graph generators that had various design principles.

A comparison of 4 RDF benchmarks (namely TPC-H~\cite{TPC-H} data expressed in RDF, LUBM, BSBM, and SP$^2$Bench) and 6 real-wold data sets (such as, e.g.,  DBpedia, the Barton Libraries Dataset~\cite{barton-benchmark} or
WordNet~\cite{Miller:1995:WLD:219717.219748}) has been reported by~\cite{Duan:2011:AOC:1989323.1989340}. The authors focus mainly on the  \emph{structuredness} (\emph{coherence}) of each benchmark dataset claiming that a primitive metric (e.g., the number of triples or the average in/outdegree) quantifies only some target characteristics of each dataset. With respect to a type $T$ the degree of structuredness of a dataset $D$  is based on  the regularity of instance data in $D$ in conforming to type $T$. The type system is extracted from the data set by finding the RDF triples that have property  \texttt{http://www.w3.org/1999/02/22-rdf-syntax-ns\#type} and extract type $T$ from their object. Properties of $T$ are determined as the union of all properties of type $T$. The structuredness is then expressed as a weighted sum of share of set properties of each type, whereas higher weights are assigned to types with more instances. The authors show that the structuredness of the chosen benchmarks is fixed, whereas real-world RDF datasets are belonging to the non-tested area of the spectrum. As a consequence, they introduce a new benchmark that receives as input any dataset associated with a required level of structuredness and size (smaller than the size of the original data), and exploits the input documents as a seed to produce a subset of the original data with the target structuredness and size. In addition, they show that structuredness and size mutually influence each other.

With the recent increasing momentum of streaming data, the Semantic Web community started to consider the issues and challenges of RDF streaming data. However, there is still a lot of open challenges that needs to tackled in this direction such as covering different real-world application scenarios. 