\section{Conclusion}
\label{sec:conclusion}

Graph data occur in a vast amount of distinct applications, such as biology, chemistry, physics, computer science, or social sciences, to name just a few. Graphs form one of the most complex data structures requiring specific and usually sophisticated approaches for processing and analysis. The history of graph theory,  that started from when these structures and their respective algorithms were studied can be traced back to the 18th century.

With the recent dawn of Big Data there have  been more occurrences of large scale graphs where the efficiency   of  processing methods is critical.  Approaches that work for smaller scale graphs often cannot be used, the data need to be processed in a distributed way and hence the efficiency is influenced by other aspects, such as limits of data transport. In addition, distribution of graphs, especially for highly connected cases, is a difficult task. Thus extensive testing of these methods for graphs of various sizes and structural complexity is extremely important.

The aim of this survey was to provide a thorough overview and comparison of graph data generators. We do not limit ourselves to a single application domain, but we cover the currently most popular areas of graph data processing. We believe that this wide scope provides a uniquely useful insight into state-of-the-art tools as well as open issues for both researchers and practitioners.
