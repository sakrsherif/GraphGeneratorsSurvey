\subsection{Graph Data Streaming}
\label{sec:generators_streaming}
One way for dealing with big graphs is to process them using the
\emph{streaming} mode where the data stream could consist of the edges of the
graph. In this mode, the graph processing algorithms can process the input
stream in the order it arrives while using only a limited amount of
memory~\cite{mcgregor2014graph}. 

\paragraph{S2Gen}  The streaming mode has mainly  attracted the attention of the
RDF and Semantic Web community. Thus, Phuoc et al.~\cite{le2012linked} presented
an evaluation framework for linked stream data processing engines. The framework
uses a datasets generated with the Stream Social network data Generator
(S2Gen), which 
simulates streams of user interactions (or events) in social networks 
(e.g., posts) in addition to the  user metadata such as users' profile
information, social network relationships, posts, photos and GPS information.
The data generator of this framework provides the users the flexibility to
control the characteristics of the generated stream by tanning a range of
parameters, which includes the frequency at which interactions are generated, 
limits such as the maximum number of messages per user
and week, and the correlation probabilities between the different objects (e.g.,
users) in the social network.

\paragraph{RSPLab} Tommasini et al.~\cite{tommasini2017rsplab} introduced
another framework for benchmarking RDF Stream Processing systems, RSPLab. The
Streamer component of this framework is designed to publish RDF streams from the
various existing RDF benchmarks (e.g., BSBM, LUBM) (see Section~\ref{sec:generators_LinkedData}).
In particular, the Streamer  component uses TripleWave\footnote{\url{http://streamreasoning.github.io/TripleWave/}}, an
open-source framework for publishing and sharing RDF streams on the
Web~\cite{mauri2016triplewave}.   TripleWave acts as a means for plugging-in
and combining streams from multiple Web data sources, in both push and pull mode.
