\subsection{General Graphs}
\label{sec:generators_general}

We start by focusing on approaches that have been designed for dealing with the
generation of general graph data that is not aimed at a particular application
domain. In general, such generators focus on reproducing properties observed in
real graphs regardless of their domain such as the degree distribution, the
diameter, the presence of a large connected component, a community structure or
a significantly large clustering coefficient.

%Currently, there exists a number of tools which involve a kind of
%general graph data generator, such as gtools from projects nauty and
%Traces~\cite{gtools} or the Stanford GraphBase~\cite{GraphBase}. We, however,
%focus on primarily generating/benchmarking projects targeting the Big Data
%world.


\paragraph {Preferential Attachment} Barabasi and
Albert~\cite{Barabasi99emergenceScaling} introduced a graph generation model
that relies on two main mechanisms. The first mechanism is continuously
expanding the graphs by adding new vertices. The second mechanism is to
preferentially attach the new vertices   to the nodes/regions that are already
well connected. So, in this approach, the generation of large graphs is governed
by standard, robust self-organizing mechanisms that go beyond the
characteristics of individual applications.

\paragraph {R-MAT} R-MAT (\emph{R}ecursive \emph{Mat}rix) is a procedural
synthetic graph generator which is designed to generate power-law degree
distributions~\cite{DBLP:conf/sdm/ChakrabartiZF04}.  The generator is recursive
and employs a fairly small number of parameters.  In principle, the strategy of
this generator is to achieve simple means to produce graphs that are matching
the properties of the real graphs. In particular, the design goal of R-MAT is to
produce graphs which mimics the degree distributions, imitate a community
structure and have a small diameter. R-MAT can generate weighted, directed and
bipartite graphs.


\paragraph{HPC Scalable Graph Analysis Benchmark} The HPC Scalable Graph
Analysis Benchmark~\cite{HPCgraph,Bader:2005:DIH:2099301.2099360} consists of a
weighted, directed graph that has a power-law distribution and four related
analysis techniques (namely graph construction, graph extraction with BFS,
                     classification of large vertex sets, and graph analysis
                     with betweenness centrality). The generator has the
following parameters: the number of nodes, the number of edges, and maximum weight of
an edge. It outputs a list of tuples containing identifiers of vertices of an
edge (with the direction from the first one to the second one) and weights
(positive integers  with a uniform random distribution) assigned to the edges of
the multigraph.  The algorithm of the generator is based on R-MAT.
%To reduce the data
%locality, the vertex ids are randomly permuted and the output
%tuples are shuffled. A related project from the same authors
%developed for the 9th DIMACS Shortest Paths Challenge is GTgraph~\cite{GTgraph}.
%t involves three types of graphs: (1) Erd\"{o}s-R\'{e}nyi random graphs, (2)
%different graphs from the DARPA HPCS SSCA\#2 graph theory benchmark
%(version 1.0), and (3) small-world graphs generated using the R-MAT graph
%generator.



\paragraph{GraphGen} For the purpose of testing the scalability of an indexing
technique called FG-index~\cite{Cheng:2007:FTV:1247480.1247574} on the size of the
database of graphs, their average size and average density, the authors have
also implemented a synthetic generator called
GraphGen\footnote{https://www.cse.ust.hk/graphgen/}. It relies on data generation code for associations and sequential
patterns provided by IBM\footnote{From 1996, no longer available at
\url{http://www.almaden.ibm.com/cs/projects/iis/hdb/Projects/data
mining/mining.shtml}}. GraphGen yields a collection of undirected, labeled and
connected graphs. It addresses the performance evaluation of frequent subgraph mining
algorithms and graph query processing algorithms. The result is represented as a
list of graphs, each consisting of a list of nodes along with a list of edges.


%\paragraph{Graph 500 Benchmark} The Graph 500 Benchmark~\cite{Graph500} includes
%a scalable data generator which produces weighted, undirected graph as a list of
%edge tuples containing the label of start vertex and end vertex together with a
%weight that represents data assigned to the edge. The space of vertex  labels is
%the set of integers beginning with 0. The input values required to describe the
%graph are (1) scale, i.e., the logarithm base two of the number of vertices, and
%(2) edge factor, i.e., the ratio of the graph's edge count to its vertex count
%(i.e., half the average degree of a vertex in the graph). The graph generator is
%a Kronecker generator similar to R-MAT. The data must not exhibit any locality,
%so in the final step the vertex labels and order of edges are randomly shuffled.
%The covered operations currently involve BFS; however, the authors intend to
%involve also two more types -- optimization (single source shortest path) and
%edge-oriented (maximal independent set) -- and five graph-related business areas:
%cybersecurity, medical informatics, data enrichment, social networks, and
%symbolic networks.

\paragraph{BTER} BTER (Block Two-Level
Erd\"{o}s-R\'{e}ny)~\cite{kolda2014scalable} is a graph generator based on the
creation of multiple Erd\"{o}s-R\'{e}ny graphs with different connection
probabilities  of which they are connected randomly between them. As the main feature, BTER is able
to reproduce input degree distributions and average clustering
coefficient per degree values. The generator starts by grouping the vertices
by degree $d$, and forming groups of size $d+1$ of nodes with degree $d$. Then, these
groups are assigned an internal edge probability in order to match the observed
average clustering coefficient of the nodes of such degree. Based on this
probability, for each node, the excess degree (i.e, the degree that in
expectation will not be realized internally in the group) is computed and used to connect
nodes from different groups at random. The authors describe a highly scalable
MapReduce based implementation that is capable of generating large graphs (with
billions of nodes) in a reasonable amounts of time.

\paragraph{Darwini} Darwini~\cite{edunov2016darwini} is an extension of BTER
designed to run on Vertex Centric computing frameworks like
Pregel~\cite{malewicz2010pregel} or Apache Giraph~\cite{ching2015one}, with the
additional feature that it is more accurate when reproducing the clustering
coefficient of the input graph. Instead of just focusing on the average
clustering coefficient for each degree, Darwini is able to model the clustering
coefficient distribution per degree. It achieves this by gathering the nodes
of the graph into buckets based on the expected number of closed triangles that they need
to close in order to attain the expected clustering coefficient. The latter is
sampled from the input distributions. Then, the vertices in each bucket are
connected randomly with a probability that would produce the expected
desired number of triangles for each bucket. Then, as in BTER, the excess degree
is used to connect the different buckets. The authors report that Darwini is
able to generate graphs with billions and even trillions of edges.

%\paragraph{gtools} Projects nauty and Traces~\cite{gtools} are programs for
%computing automorphism groups of graphs and digraphs~\cite{McKay201494}; they
%can also produce a canonical label. In the respective package there is also
%suite of programs called gtools which involve generators for non-isomorphic
%graphs, bipartite graphs, digraphs, and multigraphs.

\paragraph{Strengths and Weaknesses of General Graph Generators}
In general, existing general graph generators produce graphs with the following
properties: skewed degree distribution (e.g., powerlay), small diameter, large
clustering coefficient, a large largest connected component, and some degree of
community structure. Degree distribution can be typically configured while other
properties are just a result of the generation process and cannot be controlled
by any means.

This is not the case of BTER and Darwini, which besides the degree distribution,
they also allow tuning of the clustering coefficient. However, some use cases
demand for more control on the characteristics of the generated graphs.
This is the case, for instance, of benchmarking, where the underlying graph
structure has direct implications to the performance of the graph algorithms
to run. For this reason, the design of graph generators with the capability of fine
tuning the characteristics of the generated graphs still remains as an open
challenge. The questions that remain are what characteristics to tune
and which are the algorithms that depend on such characteristics.
