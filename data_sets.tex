\section{Graph Data Sets}

Besides graph data generators, many graph algorithms and systems are benchmarked
using real graph data sets, including new graph data generation techniques. In
this section we review the most widely used graph datasets in the literature and
their purpose.

%In general, there exist several types of testing graph data
%sets~\cite{Bachmaier2012}:

\paragraph{Large Scale Graph Analytics}

Large scale graph analytics systems are usually benchmarked using a combination of
real and synthetically generated graphs. Because of the rapid evolution of such
systems and their increasing capability to process larger graphs, the datasets
typically used  in the literature quickly change overtime. The most commonly
used datasets are, which can be downloaded from several graph dataset
repositories~\cite{snapnets,lawalgo}: 
\begin{itemize}
  \item Dblp~\cite{yang2015defining}: Represents a co-authorship network where
    researchers from Dblp are represented as vertices, and there exists and edge
    between them if they have published a paper together. 
  \item Amazon~\cite{yang2015defining}: Represents a product co-purchasing
    network, such that each vertex is a product and an edge between two products
    exists if a person has bought the two products. 
  \item Road networks~\cite{leskovec2009community}: Consist of a set of road
    networks from the US.  
  \item LiveJournal~\cite{yang2015defining}: A social network where vertices
    represent the users and the edges the acquitances between them.
  \item Orkut~\cite{yang2015defining}: Like LiveJournal, this is a social network
    where vertices represent persons and the edges their friendships.
  \item Twitter~\cite{kwak2010twitter}: This is a directed network representing
    the follower-followee interaction of Twitter.
  \item Friendster~\cite{yang2015defining}: A social network like LiveJournal and
    Orkut, where nodes represent persons and edges their relationships.
  \item WebUK~\cite{delis}: A web graph of the UK subdomain, where nodes represent
    websites and edges the hyperlinks between them.
  \item ClueWeb2012~\cite{clueweb}. This is a crawl of the web from 2012, where
    vertices represent websites and the edges the hyperlinks connecting them.
\end{itemize}

\paragraph{Recommender Systems }

Many recommender systems are based on graphs, either bipartite graphs or
many-to-many graphs. Such systems are usually tested on several real datasets,
many of them containing information about rating of products or other items such
as movies, books or songs. The following are the most widely used datasets for
testing recommender systems:

\begin{itemize}
\item MovieLens~\cite{movielens}: Consists of a bipartite graph
  between users and movies, where edges represent the ratings the different
  users have made to the movies they watched. 
\item Book-Crossings~\cite{ziegler2005improving}: Is a dataset with book ratings
  from users. 
\item Jester~\cite{goldberg2001eigentaste}: This dataset contains jokes (with
  their text) and ratings from users. 
\item Last.fm~\cite{hetrec}: This dataset contains the list of the top
  most listened artists per user, including the number of times the songs from
  those artists were played. Additionally, it contains connections between users
  (the social network they form), and a set of tags attached to artists that can
  be used to create content vectors.
\item Netflix~\cite{zhou2008large}: This dataset was used during the famous
  Netflix prize. It consists of a bipartite graph with movie ratings from users.
\end{itemize}

\paragraph{Reputation Algorithms}

Another application using social network data is that of assessing the degree of
reputation of a user based on their past interactions and
rankings~\cite{kamvar2003eigentrust,katz2006social,kumar2016edge}. Such
algorithms are typically evaluated on networks with explicit user rankings or
votings from other users, being the most widely used datasets the following
examples:

  \begin{itemize}
    \item Bitcoin~\cite{moore2013beware}: This dataset contains information from
      several Bitcoin exchanges, where users are able to rate other users after
      their transactions. Here, the vertices represent the users and the edges,
      which are labelled, the ratings between them.
    \item Wikipedia RFA~\cite{west2014exploiting}: This is a network extracted
      from Wikipedia, where the vertices represent users  and the
      edges, which are labelled, the votes emitted by administrators for the
      user to become an administrator.  Each vote is accompained with a text
      explaining the vote's sign. 
    \item WikiSigned~\cite{maniu2011building}: This is a network of Wikipedia
      editors where the vertices are the editors and the edges, which are
      labelled, represent the turst level between two editors.
    \item Extended Epinions~\cite{massa2007trust}: This is an extended version of
      the Epinions network which also contains the levels of distrust between the
      users, expressed by means of edges between them.
    \item Twitter Indian Elections~\cite{kagan2015using}: This network represents
      a Twitter network where vertices represent users and there is an edge
      between two users if a user mentions another one in a tweet. The edges
      are labelled with the average sentiment of one user towards another. 
  \end{itemize}


\paragraph{Graph partitioning, Clustering and Community Detection}

Many of the already discussed datasets are also typically used to test and
compare graph partitioning, clustering and community detection algorithms. One
can find a comprehensive lists of datasets for such algorithms
in~\cite{10dimacs,yang2015defining}, some of them already discussed for other
applications. Here, we summarize the most widely used in the literature:

\begin{itemize}
  \item Dblp, Amazon, LiveJournal, Orkut and Friendster~\cite{yang2015defining}
    are widely used for evaluating community detection since they provide
    information about the meta-communities they contain. For instance, in
    Livejournal, users can join groups of users talking about given topics. Such
    groups are exported as meta-communities, and similar approaches are used for
    other graphs. Community detection algorithms are then evaluated by trying to
    infer such meta-communities without prior knowledge of the user to group
    assignment.
  \item Zachary Karate Club~\cite{zachary1977information}: This graph consists
    of a small networks of members of a karate club that was dismissed and split
    into two new karate clubs. Vertices represent persons and edges friendship
    relationships. Information about what people joined each of the two new
    karate clubs is provided.
  \item PolBlogs~\cite{adamic2005political}: This is a network consisting of
    blogs talking about the US 2005 political elections, where a vertex
    represents a blog and the edges the hyperlinks between the blogs. Each blog
    has an associated label, whehter it is left or right oriented.
  \item PolBooks~\cite{10dimacs}: Is a network of books of Amazon that talk
    about politics, and edges between two books exist if they are co-purchased
    together. The books have labels of whether they are left or right oriented.
  \item Football~\cite{girvan2002network}: A network of american university
    football teams. Each node represents a football team and the edges represent
    the matches between them during the season. Each team is associated to a
    division.
\end{itemize}

\paragraph{Information Diffussion}

Information networks are studied using graph algorithms in order to understand
how information propagates. As such, there several datasets used to understand
such networks.

\begin{itemize}
  \item Higgs-Twitter~\cite{de2013anatomy}: This dataset contains the Twitter
    network before, during and after the discovery of the Higgs boson.
    Concretely, it contains the tweets, the mentions, retweets,
    followers/followees, etc.  
  \item Memetracker~\cite{leskovec2009memede2013}: Memetracker is a dataset that
    contains the quotes and phrases that appear more frequently on the entire
    news spectrum. It consists of the links of the news, the time and memes, and
    is used to understand how information spanws, evolves and dies.
\end{itemize}

 \paragraph{Semantic Web and Knowledge bases}

% - Wikipedia https://en.wikipedia.org/wiki/Wikipedia:Database_download
% - DbPedia

%\begin{itemize} 
%  \item A real-world data set which represents a graph, such as
%      the Internet Movie Database (IMDb)\footnote{\url{http://www.imdb.com/}} or
%      various data sets available in the Stanford Large Network Dataset
%      Collection~\cite{snapnets}, 
%    \item A general graphbase, such as the Open
%      Graph Archive~\cite{Bachmaier2012}, the Stanford
%    GraphBase~\cite{Knuth:1993:SGP:164984}, or the
%  GraphArchive~\cite{GraphArchive}, 
%\item A dataset dedicated to specific
%    experimental goals, such as graph partitioning and graph
%    clustering~\cite{10dimacs}, shortest paths~\cite{9dimacs}, or graph
%    coloring~\cite{coloring}, and 
%  \item A dataset devoted to matrices, such as
%the SuiteSparse Matrix Collection~\cite{SuiteSparse} or  the Matrix
%Market~\cite{MatrixMarket}.  
%\end{itemize}
